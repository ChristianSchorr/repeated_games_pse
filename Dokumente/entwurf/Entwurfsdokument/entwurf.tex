\documentclass[parskip=full,11pt]{scrartcl}

\usepackage[utf8]{inputenc}

%\title{Simulator für wiederholte Spiele}
%\author{Sebastian Feurer, Peter Koepernik, Luc Mercatoris,\\Christian Schorr, Pierre Toussing}

% section numbers in margins:
\renewcommand\sectionlinesformat[4]{\makebox[0pt][r]{#3}#4}

% header & footer
\usepackage{scrlayer-scrpage}
\lofoot{\today}
\refoot{\today}
\pagestyle{scrheadings}

\usepackage[sfdefault,light]{roboto}
\usepackage[T1]{fontenc}
\usepackage[german]{babel}
\usepackage[yyyymmdd]{datetime} % must be after babel
\renewcommand{\dateseparator}{-} % ISO8601 date format
\usepackage{hyperref}
\usepackage{bbm}
\usepackage{amsmath} % for $\text{}$
\usepackage{amssymb}
\usepackage[nameinlink]{cleveref}
\crefname{figure}{Abb}{Abb}
\usepackage[section]{placeins}
\usepackage{xcolor}
\usepackage{graphicx}
\usepackage{subfig}
\usepackage{float} % für Fließumgebungen; Platzierung H verschiebt nicht
\usepackage{multirow}
\restylefloat{figure}
\hypersetup{
	pdftitle={Pflichtenheft},
	bookmarks=true,
}
\usepackage{csquotes}

\newcommand\urlpart[2]{$\underbrace{\text{\texttt{#1}}}_{\text{#2}}$}

\usepackage{pflichtenheft}

\usepackage[nonumberlist]{glossaries}

\usepackage[T1]{fontenc}
\usepackage[scaled=0.85]{beramono}

\begin{document}
\begin{titlepage}
	\centering
	\vspace*{5cm}
	\includegraphics[width = 0.7\linewidth]{images/Logos/loop.png}\par
	{\huge\bfseries Ein Simulator für wiederholte Spiele\par}
	%\vspace{1cm}
	{\Large Entwurfsdokument\par}
	\vspace{2cm}
	{\Large\itshape Sebastian Feurer, Peter Koepernik, Luc Mercatoris,\\Christian Schorr, Pierre Toussing\par}
	\vfill
	{\large \today\par}
\end{titlepage}

%\tableofcontents
%\pagebreak

\section{Einleitung}
TODO

\section{Spezifikationsänderungen}

\subsection{Populationen}
Das Konzept von Gruppen und Segmenten wurde überarbeitet. Die Gesamtheit der Gruppen, Segmente, sowie deren Größen und Einstellungen (bis auf Multikonfiguration) heißt nun eine \enquote{Population}. Die Konfiguration von Gruppen und Populationen wurde aus dem Konfigurationsfenster ausgelagert. Wie zur Strategie- und Stufenspielerstellung gibt es nun Fenster zur Erstellung einzelner Gruppen und Populationen.

\textbf{Gruppenerstellung:}
Im Gruppenerstellungsfenster können ein Name und eine Beschreibung der Gruppe eingetragen werden. Weiter können Anzahl und Größe von Segmenten sowie jeweils Strategie- und Kapitalverteilung eingestellt werden. Weiterhin kann eine Gruppe per Checkbox als \enquote{kohärent} deklariert werden (siehe \cref{?}). Diese Checkbox ist voreingestellt aktiviert. Wird sie deaktiviert, so identifizieren sich Agenten dieser Gruppe gegenseitig nicht mehr als Agenten derselben Gruppe. Diese Funktionalität realisiert das Konzept der \enquote{gruppenlosen Agenten}. Eine fertig konfigurierte Gruppe kann als Datei exportiert und eine als Datei exportierte Gruppe kann importiert werden.

\textbf{Populationserstellung:}
Im Populationserstellungsfenster können ein Name sowie eine Beschreibung der Population eingetragen werden. Dann können \(1\) bis \(16\) der selbst erstellten oder von vornherein im Programm hinterlegten Gruppen hinzugefügt werden. Für jede Gruppe können die Anzahl der Mitglieder über ein Textfeld eingegeben werden (siehe \cref{?}). Eine fertig konfigurierte Population kann als Datei exportiert und eine als Datei exportierte Population kann importiert werden.

Das Konfigurationsfenster ist nun nicht mehr in drei Bereiche unterteilt. Statt der Gruppeneinstellungen befindet sich nun zwischen den nun verschmolzenen, ehemaligen \enquote{Grund-} und \enquote{Erweiterten Einstellungen} nur noch ein Dropdown-Menü mit der Betitelung \enquote{Population} (siehe \cref{?}). Hier kann eine der selbst erstellten oder von vornherein im Programm hinterlegten Populationen ausgewählt werden. Unter dem Dropdown-Menü kann ein Abschnitt ausgeklappt werden, der die Zusammensetzung der Population zeigt (siehe \cref{?}). Dieser hat dieselbe Form wie die ehemaligen Gruppeneinstellungen, mit dem Unterschied, dass keine Modifikationen mehr vorgenommen werden können. Per Checkbox kann dort die \enquote{Multikonfiguration} für Segmente und Gruppen wie zuvor aktiviert werden.

\subsection{Strategien}
Die Liste aller Variablen, die in einer Strategie als Literal vorkommen können wurde um die folgenden erweitert:

\begin{itemize}
\item B hat bei bisherigen Spielen zwischen B und Agenten derselben Gruppenzugehörigkeit wie A (im aktuellen Adaptionsschritt) immer/nie/letztes Mal/wenigstens einmal kooperiert
\item B hat bei bisherigen Spielen zwischen B und Agenten mit ähnlichem aktuellen Kapital wie A (im aktuellen Adaptionsschritt) immer/nie/letztes Mal/wenigstens einmal kooperiert
\end{itemize}

\section{Pakete und Klassen}
\subsection{Paket \texttt{edu.kit.loop.model}}
Das Modell beinhaltet Klassen und Methoden zum Starten und Abbrechen von Simulationen, sowie zum Erstellen und Speichern von Konfigurationen, Stufenspielen, Strategien und Populationen.

\subsubsection{Class \texttt{UserConfiguration}}
Implements: \texttt{java.io.Serializable}

Diese Klasse repräsentiert eine vom Nutzer erstellte Konfiguration. Sie bietet Methoden zum Lesen aller zugehörigen Parameter.

Konstruktoren:
\begin{itemize}\itemsep -10pt
\item \texttt{UserConfiguration(String gameName, int agentCount, int roundCount, int iterationCount, List<String> availableStrategyNames, boolean mixedAllowed, List<String> populationNames, String pairBuilderName, List<double> pairBuilderParameters, String successQuantifierName, List<double> successQuantifierName, String strategyAdjusterName, List<double> strategyAdjusterParameters, String equilibriumCriterionName, List<double> equilibriumCriterionParameters, int maxAdapts, boolean isMulticonfiguration, String variableParameterName, double startValue, double, endValue, double stepSize)}
\item[] Creates a new \texttt{UserConfiguration} with the given parameters
\item[] \texttt{gameName}: the name of the game
\item[] \texttt{agentCount}: the amount of agents
\item[] \texttt{roundCount}: the amount of rounds per adaptionstep
\item[] \texttt{iterationCount}: the amount of iterations
\item[] \texttt{availableStrategyNames}: a list with the names of all allowed strategies
\item[] \texttt{mixedAllowed}: \texttt{true}, if mixed strategies are allowed, \texttt{false} otherwise.
\item[] \texttt{populationNames}: a list with the names of all populations
\item[] \texttt{pairBuilderNames}: the name of the pair builder
\item[] \texttt{pairBuilderParameters}: a list with the values of the parameters of the pair builder
\item[] \texttt{successQuantifierNames}: the name of the success quantifier
\item[] \texttt{successQuantifierParameters}: a list with the values of the parameters of the success quantifier
\item[] \texttt{strategyAdjusterName}: the name of the strategy adjuster
\item[] \texttt{strategyAdjusterParameters}: a list with the values of the parameters of the strategy adjuster
\item[] \texttt{equilibriumCriterionName}: the name of the equilibrium criterion
\item[] \texttt{equilibriumCriterionParameters}: a list with the values of the parameters of the equilibrium criterion
\item[] \texttt{maxAdapts}: the maximum amount of simulated adaption steps per iteration
\item[] \texttt{isMulticonfiguration}: \texttt{true}, if this is a multiconfiguration, \texttt{false} otherwise
\item[] \texttt{variableParameterName}: the name of the multiconfiguration parameter, if this is a multiconfiguration, \texttt{""} otherwise
\item[] \texttt{startValue}: the starting value of the multiconfiguration parameter, if this is a multiconfiguration, \(0\) otherwise
\item[] \texttt{endValue}: the end value of the multiconfiguration parameter, if this is a multiconfiguration, \(0\) otherwise
\item[] \texttt{stepSize}: the step size of the multiconfiguration parameter, if this is a multiconfiguration, \(0\) otherwise
\end{itemize}

Methoden:
\begin{itemize}\itemsep -10pt
	\item \texttt{String getGameName()}
	\item[] Returns the name of the game of this configuration
	\item[] Returns: the name of the game of this configuration
	
	\item \texttt{int getAgentCount()}
	\item[] Returns the amount of agents in this configuration
	\item[] Returns: the amount of agents in this configuration
	
	\item \texttt{int getRoundCount()}
	\item[] Returns the amount of rounds per adaption step in this configuration
	\item[] Returns: the amount of rounds per adaption step in this configuration
	
	\item \texttt{int getIterationCount()}
	\item[] Returns the amount of iterations in this configuration
	\item[] Returns: the amount of iterations in this configuration
	
	\item \texttt{List<String> getAvailableStrategyNames()}
	\item[] Returns a list with names of all available strategies in this configuration
	\item[] Returns: a list with names of all available strategies in this configuration
	
	\item \texttt{boolean getMixedAllowed()}
	\item[] Returns, whether mixed strategies are allowed in this configuration
	\item[] Returns: \texttt{true}, if mixed strategies are allowed in this configuration; \texttt{false} otherwise
	
	\item \texttt{List<String> getPopulationNames()}
	\item[] Returns a list with the names of all populations in this configuration
	\item[] Returns: a list with the names of all populations in this configuration
	
	\item \texttt{String getPairBuilderName()}
	\item[] Returns the name of the \texttt{PairBuilder} of this configuration
	\item[] Returns: the name of the \texttt{PairBuilder} of this configuration
	
	\item \texttt{List<double> getPairBuilderParameters()}
	\item[] Returns a list with the values of the parameters of the \texttt{PairBuilder} of this configuration
	\item[] Returns: a list with the values of the parameters of the \texttt{PairBuilder} of this configuration
	
	\item \texttt{String getSuccessQuantifierName()}
	\item[] Returns the name of the \texttt{SuccessQuantifier} of this configuration
	\item[] Returns: the name of the \texttt{SuccessQuantifier} of this configuration
	
	\item \texttt{List<double> getSuccessQuantifierParameters()}
	\item[] Returns a list with the values of the parameters of the \texttt{SuccessQuantifier} of this configuration
	\item[] Returns: a list with the values of the parameters of the \texttt{SuccessQuantifier} of this configuration
	
	\item \texttt{String getStrategyAdjusterName()}
	\item[] Returns the name of the \texttt{PairBuilder} of this configuration.
	\item[] Returns: the name of the \texttt{PairBuilder} of this configuration
	
	\item \texttt{List<double> getStrategyAdjusterParameters()}
	\item[] Returns a list with the values of the parameters of the \texttt{StrategyAdjuster} of this configuration
	\item[] Returns: a list with the values of the parameters of the \texttt{StrategyAdjuster} of this configuration
	
	
	\item \texttt{String getEquilibriumCriterionName()}
	\item[] Returns the name of the \texttt{EquilibriumCriterion} of this configuration.
	\item[] Returns: the name of the \texttt{EquilibriumCriterion} of this configuration.
	
	\item \texttt{List<double> getEquilibriumParameters()}
	\item[] Returns a list with the values of the parameters of the \texttt{EquilibriumCriterion} of this configuration
	\item[] Returns: a list with the values of the parameters of the \texttt{EquilibriumCriterion} of this configuration
	
	\item \texttt{int getMaxAdapts()}
	\item[] Returns the maximum amount of simulated adaption steps per iteration in this configuration
	\item[] Returns: the maximum amount of simulated adaption steps per iteration in this configuration
	
	\item \texttt{boolean isMulticonfiguration()}
	\item[] Returns whether this is a multiconfiguration.
	\item[] Returns: \texttt{true}, if this is a multiconfiguration, \texttt{false} otherwise
	
	\item \texttt{String getVariableParameterName()}
	\item[] Returns the name of the multiconfiguration parameter of this configuration, if this is a multiconfiguration; \texttt{""} otherwise.
	\item[] Returns: the name of the multiconfiguration parameter of this configuration, if this is a multiconfiguration; \texttt{""} otherwise
	
	\item \texttt{double getStartValue()}
	\item[] Returns the starting value of the multiconfiguration parameter of this configuration, if this is a multiconfiguration; \(0\) otherwise.
	\item[] Returns: the starting value of the multiconfiguration parameter of this configuration, if this is a multiconfiguration; \(0\) otherwise
	
	\item \texttt{double getEndValue()}
	\item[] Returns the end value of the multiconfiguration parameter of this configuration, if this is a multiconfiguration; \(0\) otherwise.
	\item[] Returns: the end value of the multiconfiguration parameter of this configuration, if this is a multiconfiguration; \(0\) otherwise
	
	\item \texttt{double getStepSize()}
	\item[] Returns the step size of the multiconfiguration parameter of this configuration, if this is a multiconfiguration; \(0\) otherwise.
	\item[] Returns: the step size of the multiconfiguration parameter of this configuration, if this is a multiconfiguration; \(0\) otherwise
	
	
\end{itemize}


\subsubsection{Class \texttt{Configuration}}
Diese Klasse repräsentiert die elementare Konfiguration einer einzelnen Wiederholung und enthält alle Informationen zum Start einer solchen:
\begin{itemize}\itemsep -10pt
\item Stufenspiel
\item Anzahl von Agenten
\item Runden pro Adaptionsschritt
\item Ob gemischte Strategien zugelassen sind
\item Gruppen-/Segmenteinteilungen
\item Kapital- und Strategieinitialisierung der Segmente
\item Agentenpaarung
\item Erfolgsquantifizierung
\item Adaptionsmechanismus
\item Gleichgewichtskriterium
\item Maximale Zahl von Adaptionsschritten
\end{itemize}

Konstruktoren:
\begin{itemize}\itemsep -10pt
\item \texttt{Configuration(Game game, int roundCount, boolean mixedStrategies, Collection<Segment> segments, PairBuilder pairBuilder, SuccessQuantifier successQuantifier, SrategyAdjuster strategyAdjuster, EquilibriumCriterion equilibriumCriterion, int maxAdapts)}
\item[] Creates a new \texttt{Configuration} with the given parameters.
\item[] \texttt{game}: the game of thios configuration
\item[] \texttt{roundCount}: the amount of rounds per adaption step in this configuration
\item[] \texttt{mixedStrategies}: \texttt{true}, if mixed strategies are allowed in this configuration, \texttt{false} otherwise
\item[] \texttt{segments}: all segments of this configuration
\item[] \texttt{pairBuilder}: the \texttt{PairBuilder} of this configuration
\item[] \texttt{successQuantifier}: the \texttt{SuccessQuantifier} of this configuration
\item[] \texttt{strategyAdjuster}: the \texttt{StrategyAdjuster} of this configuration
\item[] \texttt{equilibriumCriterion}: the \texttt{EquilibriumCriterion} of this configuration
\item[] \texttt{maxAdapts}: the maximum amount of simulated adaption steps per iteration in this configuration
\end{itemize}

Methoden:
\begin{itemize}\itemsep -10pt
\item \texttt{Game getGame()}
\item[] Returns the game of this configuration
\item[] Returns: the game of this configuration

\item \texttt{int getAgentCount()}
\item[] Returns the amount of agents in this configuration
\item[] Returns: the amount of agents in this configuration

\item \texttt{int getRoundCount()}
\item[] Returns the amount of rounds per adaption step in this configuration
\item[] Returns: the amount of rounds per adaption step in this configuration

\item \texttt{boolean allowsMixedStrategies()}
\item[] Returns, whether mixed strategies are allowed in this configuration.
\item[] Returns: \texttt{true}, if mixed strategies are allowed in this configuration, \texttt{false} otherwise

\item \texttt{Collection<Segment> getSegments()}
\item[] Returns all segments belonging to this configuration
\item[] Returns: all segments belonging to this configuration as \texttt{Collection<Segment>}

\item \texttt{PairBuilder getPairBuilder()}
\item[] Returns the \texttt{pairBuilder} of this configuration.
\item[] Returns: the \texttt{pairBuilder} of this configuration

\item \texttt{SuccessQuantifier getSuccessQuantifier()}
\item[] Returns the \texttt{SuccessQuantifier} of this configuration.
\item[] Returns: the \texttt{SuccessQuantifier} of this configuration

\item \texttt{StrategyAdjuster getStrategyAdjuster()}
\item[] Returns the \texttt{StrategyAdjuster} of this configuration.
\item[] Returns: the \texttt{StrategyAdjuster} of this configuration

\item \texttt{EquilibriumCriterion getEquilibriumCriterion()}
\item[] Returns the \texttt{EquilibriumCriterion} of this configuration.
\item[] Returns: the \texttt{EquilibriumCriterion} of this configuration

\item \texttt{int getMaxAdapts()}
\item[] Returns the maximum amount of simulated adaption steps per iteration in this configuration
\item[] Returns: the maximum amount of simulated adaption steps per iteration in this configuration
\end{itemize}

\subsubsection{Interface \texttt{Nameable}}

Implementierungen dieses Interface haben einen Namen und eine Beschreibung, die in Form von \texttt{String}s abgefragt werden können.

Methoden:
\begin{itemize}\itemsep -10pt
\item \texttt{String getName()}
\item[] Returns the name of this object.
\item[] Returns: the name of this object

\item \texttt{String getDescription()}
\item[] Returns the description of this object.
\item[] Returns: the description of this object
\end{itemize}

\subsubsection{Class \texttt{Population}}
Implements: \texttt{Nameable}

Diese Klasse repräsentiert eine Population. Sie bietet Methoden zur Abfrage von Größe und Gruppenzusammensetzung.

Konstruktoren:
\begin{itemize}\itemsep -10pt
\item \texttt{Population(String name, String description, List<Group> groups, List<Integer> groupSizes)}
\item[] Creates a new \texttt{Population} with name, description and \texttt{Group} composition as given.
\item[] \texttt{name}: the name of this population
\item[] \texttt{description}: the description of this population
\item[] \texttt{groups}: the groups this population is composed of
\item[] \texttt{groupSizes}: the sizes of the groups, in the same order as the groups themselves
\end{itemize}

Methoden:
\begin{itemize}\itemsep -10pt
\item \texttt{int getSize()}
\item[] Returns the size of this population.
\item[] Returns: the size of this population

\item \texttt{List<Group> getGroups()}
\item[] Returns the groups this population is composed of.
\item[] Returns: the groups this population is composed of

\item \texttt{int getGroupSize(Group group)}
\item[] Returns the size of the given group if it is part of this population, \(0\) otherwise.
\item[] \texttt{group}: the group, whose size shall be returned
\item[] Returns: the size of the given group if it is part of this population, \(0\) otherwise

\item \texttt{int getGroupCount()}
\item[] Returns the amount of groups in this population
\item[] Returns: the amount of groups in this population
\end{itemize}

\subsubsection{Class \texttt{Group}}
Implements: \texttt{Nameable}

Diese Klasse repräsentiert eine Gruppe. Sie bietet Methoden zur Abfrage von Segmentzusammensetzung und Kohäsion.

Konstruktoren:
\begin{itemize}\itemsep -10pt
\item \texttt{Group(String name, String description, List<Segment> segments, List<Double> segmentSizes, boolean isCohesive)}
\item[] Creates a new \texttt{Group} with name, description and \texttt{Segment} composition as given
\item[] \texttt{name}: the name of this group
\item[] \texttt{description}: the description of this group
\item[] \texttt{segments}: the segments this group is composed of
\item[] \texttt{segmentSizes}: the relative sizes of the segments in the same order as the segments themselves
\item[] \texttt{isCohesive}: indicates whether this group is cohesive
\end{itemize}

Methoden:
\begin{itemize}\itemsep -10pt
\item \texttt{List<Segment> getSegments()}
\item[] Returns a list of the segments this group is composed of.
\item[] Returns: a list of the segments this group is composed of

\item \texttt{double getSegmentSize(Segment segment)}
\item[] Returns the relative size of the given \texttt{Segment} if it is part of this group, \(0\) otherwise.
\item[] \texttt{segment}: the segment whose relative size shall be returned
\item[] Returns: the relative size of the given \texttt{Segment} if it is part of this group, \(0\) otherwise

\item \texttt{int getSegmentCount()}
\item[] Returns the amount of segments in this group.
\item[] Returns: the amount of segments in this group

\item \texttt{boolean isCohesive()}
\item[] Returns whether this group is cohesive.
\item[] Returns: \texttt{true} if this group is cohesive, \texttt{false} otherwise
\end{itemize}

\subsubsection{Class \texttt{Segment}}

Diese Klasse repräsentiert ein Segment. Es bietet Methoden zur Abfrage von Kapital- und Strategieverteilung. Dabei werden lediglich die Namen der Kapitalverteilung sowie der verschiedenen Strategien gespeichert, die zu der entsprechenden \texttt{DiscreteDistribution} bzw. den entsprechenden \texttt{Strategy}s im zentralen Repository korrespondieren.

Konstruktoren:
\begin{itemize}\itemsep -10pt
\item \texttt{Segment(String capitalDistributionName, Collection<String> strategyNames)}
\item[] Creates a new \texttt{Segment} with the given capital- and strategy distribution
\item[] \texttt{capitalDistributionName}: the name of the capital distribution of this segment
\item[] \texttt{strategyNames}: the names of the strategies in the strategy distribution of this segment
\end{itemize}

Methoden:
\begin{itemize}\itemsep -10pt
\item \texttt{String getCapitalDistributionName()}
\item[] Returns the name of the capital distribution of this segment.
\item[] Returns: the name of the capital distribution of this segment

\item \texttt{Collection<String> getStrategyNames()}
\item[] Returns the names of the strategies in the strategy distribution of this segment.
\item[] Returns: the names of the strategies in the strategy distribution of this segment
\end{itemize}


\subsection{Paket \texttt{edu.kit.loop.model.simulator}}
Dieses Paket enthält das Interface \texttt{Simulator}. Dieses bietet eine Schnittstelle zum Starten und Abbrechen von Simulationen. Beim Start einer Simulation wird eine Referenz auf ein \texttt{Simulation}-Objekt zurückgegeben, über das der Ausführungsstatus und die Ergebnisse der Simulation abgefragt werden können.

\subsubsection{Class \texttt{Simulation}}
Ein \texttt{Simulation}-Objekt enthält Informationen zu einer gestarteten Simulation, etwa deren Konfiguration, Ausführungsstatus, \texttt{id} und gegebenenfalls die Ergebnisse der Simulation. Es wird von einem \texttt{Simulator} erzeugt und bereitgestellt, wenn eine Simulation gestartet wird.

Die Klasse unterscheidet nicht zwischen Multikonfigurationen und Nicht-Multikonfigurationen, geht also allgemein von mehreren zugrundeliegenden elementaren Konfigurationen aus.

Konstruktoren:
\begin{itemize} \itemsep -10pt
\item \texttt{Simulation(UserConfiguration config, int id)}
\item[] Creates a new \texttt{Simulation} with the given \texttt{UserConfiguration} and \texttt{id}.
\item[] \texttt{config}: the configuration of this simulation
\item[] \texttt{id}: the \texttt{id} of this simulation
\end{itemize}

Methoden:
\begin{itemize}\itemsep -10pt
\item \texttt{protected void addIterationResult(IterationResult result, int i)}
\item[] adds an \texttt{IterationResult} to the \texttt{i}-th elementary configuration of this simulation
\item[] \texttt{result}: the \texttt{IterationResult} that shall be added
\item[] \texttt{i}: the elementary configuration to which the given \texttt{IterationResult} shall be added

\item \texttt{void registerIterationFinished(Consumer<IterationResult> action)}
\item[] Registers an action that will be executed every time an iteration of this simulation is finished. The \texttt{IterationResult} of the iteration will be passed as an argument to the action
\item[] \texttt{action}: the action that shall be executed whenever an iteration finishes

\item \texttt{List<IterationResult> getIterationResults(int i)}
\item[] Returns a list of all yet available \texttt{IterationResult}s of iterations with the \texttt{i}-th elementary configuration.
\item[] \texttt{i}: the elementary configuration, whose finished iterations shall be returned
\item[] Returns: a list of all yet available \texttt{IterationResult}s of iterations with the \texttt{i}-th elementary configuration

\item \texttt{UserConfiguration getUserConfiguration()}
\item[] Returns the \texttt{UserConfiguration} of this simulation.
\item[] Returns: the \texttt{UserConfiguration} of this simulation

\item \texttt{int getConfigurationCount()}
\item[] Returns the amount of elementary configurations of this simulation (\(1\) if this is not a multiconfiguration).
\item[] Returns: the amount of elementary configurations of this simulation

\item \texttt{int getId()}
\item[] Returns the \texttt{id} of this simulation.
\item[] Returns: the \texttt{id} of this simulation
\end{itemize}

\subsubsection{Interface \texttt{Simulator}}
Über einen Simulator können Simulationen gestartet und abgebrochen werden. Zum Starten einer Simulation muss dem \texttt{Simulator} eine \texttt{UserConfiguration} übergeben werden, die die durchzuführende Simulation spezifiziert. Daraufhin wird ein \texttt{Simulation}-Objekt erzeugt und zurückgegeben, über das der Ausführungsstatus und die Ergebnisse der gestarteten Simulation abgefragt werden können. Jeder Simulation wird beim Start eine eindeutige \texttt{id} zugewiesen.

Methoden:
\begin{itemize} \itemsep -10pt
\item \texttt{Simulation startSimulation(UserConfiguration config)}
\item[] Starts a new simulation with the given \texttt{UserConfiguration} and returns a handle to a \texttt{Simulation}-object for the started simulation.
\item[]\texttt{config}: the configuration for which a simulation shall be started
\item[] Returns: a handle to a \texttt{Simulation}-object for the started simulation

\item \texttt{Simulation startSimulation(UserConfiguration config, Consumer<Simulation> action)}
\item[] Starts a new simulation with the given \texttt{UserConfiguration} and returns a handle to a \texttt{Simulation}-object for the started simulation. Executes the given action with the \texttt{Simulation}-object passed as parameter when the started simulation is finished.
\item[] \texttt{config}: the \texttt{UserConfiguration} a new simulation shall be started with
\item[] \texttt{action}: the action that shall be executed when the simulation is finished
\item[] Returns: a handle to a \texttt{Simulation}-object for the started simulation

\item \texttt{boolean stopSimulation(Simulation sim)}
\item[] Stop the execution of the given simulation if it is currently running.
\item[] \texttt{sim}: the simulation whose execution shall be stopped
\item[] Returns: \texttt{true}, if the execution of the simulation was successfully stopped, \texttt{false} otherwise

\item \texttt{boolean stopSimulation(int id)}
\item[] If a simulation with the given \texttt{id} is currently running, stop its execution.
\item[] \texttt{id}: the \texttt{id} of the simulation that shall be stopped
\item[] Returns: \texttt{true}, if the execution of the simulation was successfully stopped, \texttt{false} otherwise

\item \texttt{void stopAllSimulation()}
\item[] Stops the execution of all running simulations.

\item \texttt{Simulation getSimulation(int id)}
\item[] Returns the \texttt{Simulation}-object of the simulation with the given \texttt{id}, if existent; \texttt{null} otherwise
Gibt das \texttt{Simulation}-Objekt der Simulation mit der entsprechenden \texttt{id} zurück, falls existent. Ansonsten \texttt{null}.
\item[] \texttt{id}: the \texttt{id} of the simulation, whose \texttt{Simulation}-object shall be returned
\item[] Returns: the \texttt{Simulation}-object of the simulation with the given \texttt{id}, if existent; \texttt{null} otherwise
\end{itemize}

\subsubsection{Class \texttt{ThreadPoolSimulator}}
Implements: \texttt{Simulator}

Eine Implementierung des \texttt{Simulator}-Interfaces. Führt die Wiederholungen parallel in einem \texttt{ThreadPool} aus.

Konstruktoren:
\begin{itemize}\itemsep -10pt
\item \texttt{ThreadPoolSimulator()}
\item[] Creates a new \texttt{ThreadPoolSimulator}.

\item \texttt{ThreadPoolSimulator(int maxThreads)}
\item[] Creates a new \texttt{ThreadPoolSimulator} with the given maximum amount of running \texttt{Thread}s.
\item[] \texttt{maxThreads}: the maximum amount of running \texttt{Thread}s in the \texttt{ThreadPool}
\end{itemize}

Methoden:
\begin{itemize}\itemsep -10pt
\item \texttt{int getRunningIterationCount()}
\item[] Returns the amount of currently executed iterations.
\item[] Returns: the amount of currently executed iterations

\item \texttt{int getQueuedIterationCount()}
\item[] Returns the amount of iterations currently waiting for execution.
\item[] Returns: the amount of iterations currently waiting for execution
\end{itemize}

\subsubsection{Class \texttt{ConfigurationCreator}}
Diese Klasse nimmt eine \texttt{UserConfiguration} entgegen und extrahiert daraus alle zugehörigen elementaren Konfigurationen. Diese werden als \texttt{Configuration}s zurückgegeben.

Konstruktoren:
\begin{itemize}\itemsep -10pt
\item \texttt{ConfigurationCreator()}
\item[] Creates a new \texttt{ConfigurationCreator}.
\end{itemize}

Methoden:
\begin{itemize}\itemsep -10pt
\item \texttt{List<Configuration> generateConfigurations(UserConfiguration config)}
\item[] Generates all associated elementary configurations of the given \texttt{UserConfiguration} and returns them as \texttt{Configuration}s.
\item[] \texttt{config}: the \texttt{UserConfiguration} whose associated elementary configurations shall be generated
\item[] Returns: all associated elementary configurations of the given \texttt{UserConfiguration} as \texttt{Configuration}s
\end{itemize}

\subsection{Paket \texttt{edu.kit.loop.model.simulationengine}}

Dieses Paket beinhaltet die Klasse \texttt{SimulationEngine}, mit der eine einzelne Wiederholung zu einer bestimmten elementaren Konfiguration ausgeführt werden kann, sowie für den Simulationsablauf zentrale Klassen wie den \texttt{Agent}, das Stufenspiel (\texttt{Game}) oder die \texttt{Strategy}.

\subsubsection{Class \texttt{AgentInitializer}}
Ein \texttt{AgentInitializer} kann verwendet werden, um die Agenten eines gegebenen Segments zu erstellen und mit Kapital, Strategien und Gruppenzugehörigkeit zu initialisieren.

Konstruktoren:
\begin{itemize}\itemsep -10pt
\item \texttt{AgentInitializer()}
\item[] Creates a new \texttt{AgentInitializer}.
\end{itemize}

Methoden:
\begin{itemize}\itemsep -10pt
\item \texttt{Collection<Agent> initializeAgents(Segment segment)}
\item[] Creates and returns all agents of the given segment and initialises them with capital, strategy and group affiliation.
\item[] Returns: the initialised agents
\end{itemize}

\subsubsection{Interface \texttt{SimulationHistory}}
Eine Implementierung des \texttt{SimulationHistory}-Interfaces speichert für jeden Agenten die Ergebnisse aller bisherigen Runden eines Adaptionsschrittes.

Methoden:
\begin{itemize}\itemsep -10pt
	\item \texttt{void addResult(GameResult result)}
	\item[] Adds the result of (the game of) a round to the history.
	\item[] \texttt{result}: the \texttt{GameResult} that shall be added
	\item \texttt{List<GameResult> getAllResults()}
	\item[] Returns a list of the \texttt{GameResult}s of all agents and all rounds of the current adaption step.
	\item[]Returns: a list of the \texttt{GameResult}s of all agents and all rounds of the current adaption step
	\item \texttt{List<GameResult> getResultsByAgent(agent: Agent)}
	\item[] Returns a list of all \texttt{GameResult}s of the given agent.
	\item[] \texttt{agent}: the agent whose results shall be returned
	\item[]Returns: a list of all \texttt{GameResult}s of the given agent
	\item \texttt{List<GameResult> getLatestResults()}
	\item[] Returns the most recent \texttt{GameResult} of every agent.
	\item[]Returns: the most recent \texttt{GameResult} of every agent
	\item \texttt{GameResult getLatestResultByAgent(agent: Agent)}
	\item[] Returns the most recent \texttt{GameResult} of the given agent.
	\item[] \texttt{agent}: the agent whose latest result shall be returned
	\item[]Returns: the most recent \texttt{GameResult} of the given agent
	\item \texttt{List<GameResult> getAllWhere(condition: Predicate<GameResult>)}
	\item[] Filters all \texttt{GameResult} with the given condition and returns them.
	\item[]\texttt{condition}: the filter condition as \texttt{Predicate<GameResult>}
	\item[]Returns: Die filtered \texttt{GameResult}s
	\item \texttt{GameResult getLatestWhere(condition: Predicate<GameResult>)}
	\item[] Returns the latest \texttt{GameResult} that meets the given condition
	\item[]\texttt{condition}: the filter condition as \texttt{Predicate<GameResult>}
	\item[]Returns: the latest \texttt{GameResult} that meets the given condition
\end{itemize}

\subsubsection{Class \texttt{SimulationHistoryTable}}
Implements: \texttt{SimulationHistory}

Diese Klasse implementiert das \texttt{SimulationHistory}-Interface.

\subsubsection{Class \texttt{SimulationEngine}}

Diese Klasse bietet eine Methode zur Ausführung einer einzelnen Wiederholung zu einer gegebenen elementaren Konfiguration. Das Ergebnis wird in Form eines \texttt{IterationResult}-Objekts zurückgegeben.

Methoden:
\begin{itemize}\itemsep -10pt
\item \texttt{IterationResult executeIteration(Configuration configuration)}
\item[] Executes an iteration with the given elementary configuration and returns an \texttt{IterationResult}-object upon finishing.
\item[] \texttt{configuration}: the elementary configuration of the iteration that shall be executed
\item[] Returns: the result of the iteration as \texttt{IterationResult}
\end{itemize}

\subsubsection{Class \texttt{IterationResult}}
In einem Objekt dieser Klasse können die Ergebnisse einer durchgeführten Wiederholung gespeichert und abgefragt werden:
\begin{itemize}\itemsep -10pt
\item Die Agenten, geordnet nach Rang als \texttt{List<Agent>}
\item Der Verlauf des letzten Adaptionsschritts als \texttt{SimulationHistory}
\item Ob ein Gleichgewicht eingetreten ist
\item Die Effizienz des finalen Zustands
\item Die Zahl der durchgeführten Adaptionsschritte
\end{itemize}

Konstruktoren:
\begin{itemize}\itemsep -10pt
\item \texttt{IterationResult(List<Agent> agents, SimulationHistory history, boolean equilibriumReached, double efficiency, int adapts)}
\item[] Creates a new \texttt{IterationResult}-object with the given iteration results.
\item[] \texttt{agents}: the agents, ordered by final rank
\item[] \texttt{history}: the \texttt{SimulationHistory} of the last adaption step
\item[] \texttt{equilibriumReached}: \texttt{true} if an equilibrium was reached, \texttt{false} otherwise
\item[] \texttt{efficiency}: the efficiency of the final state
\item[] \texttt{adapts}: the amount of simulated adaption steps
\end{itemize}

Methoden
\begin{itemize}\itemsep -10pt
\item \texttt{List<Agent> getAgents()}
\item[] Returns a list of all agents, ordered by final rank
\item[] Returns: a list of all agents, ordered by final rank

\item \texttt{SimulationHistory getHistory()}
\item[] Returns the \texttt{SimulationHistory} of the last adaption step
\item[] Returns: the \texttt{SimulationHistory} of the last adaption step

\item \texttt{boolean equilibriumReached()}
\item[] Returns whether an equilibrium was reached
\item[] Returns: \texttt{true} if an equilibrium was reached, \texttt{false} otherwise

\item \texttt{double getEfficiency()}
\item[] Returns the efficiency of the final state
\item[] Returns: the efficiency of the final state

\item \texttt{int getAdapts()}
\item[] Returns the amount of simulated adaption steps
\item[] Returns: the amount of simulated adaption steps
\end{itemize}

\subsubsection{Class \texttt{Agent}}
Diese Klasse repräsentiert einen einzelnen Agenten in einem Simulationsdurchlauf. Sie enthält Informationen über den Agenten wie dessen Gruppenzugehörigkeit, sein aktuelles und initiales Kapital oder seine aktuelle Strategie.

Konstruktoren:
\begin{itemize}\itemsep -10pt
\item \texttt{Agent(int initialCapital, Strategy initialStrategy, int groupId)}
\item[] Creates a new \texttt{Agent} with given initial capital, given strategy and given group affiliation
\item[] \texttt{initialCaptial}: the initial captital of the agent
\item[] \texttt{initialStrategy}: the initial \texttt{Strategy} of the agent
\item[] \texttt{groupId}: the \texttt{id} of the group this agent belongs to, if it is cohesive; \(-1\) otherwise
\end{itemize}

Methoden:
\begin{itemize}\itemsep -10pt
\item \texttt{int getCapital()}
\item[] Returns the current capital of this agent.
\item[] Returns: the current capital of this agent

\item \texttt{int getInitialCapital()}
\item[] Returns the initial capital of this agent.
\item[] Returns: the initial capital of this agent

\item \texttt{void addCapital(int capital)}
\item[] Increases the captial of this agent by the given amount.

\item \texttt{Strategy getStrategy()}
\item[] Returns the current \texttt{Strategy} of this agent
\item[] Returns: the current \texttt{Strategy} of this agent

\item\texttt{void setStrategy(Strategy strategy)}
\item[] Sets the \texttt{Strategy} of this agent to the given one	.

\item \texttt{int getGroupId()}
\item[] Returns the \texttt{id} of the group this agent belongs to, if it is cohesive; \(-1\) otherwise.
\item[] Returns: the \texttt{id} of the group this agent belongs to, if it is cohesive; \(-1\) otherwise
\end{itemize}

\subsubsection{Interface \texttt{AgentPair}}
Dieses Interface repräsentiert ein Paar von Agenten.

Methoden:
\begin{itemize}\itemsep -10pt
\item \texttt{getFirstAgent()}
\item[] Returns the first of both agents.
\item[] Returns: the first of both agents

\item \texttt{getSecondAgent()}
\item[] Returns the second of both agents.
\item[] Returns: the second of both agents
\end{itemize}

\subsubsection{Class \texttt{GameResult}}

Diese Klasse speichert das Ergebnis eines Spiels zwischen zwei Agenten. Das Ergebnis besteht aus:
\begin{itemize}\itemsep -10pt
	\item Den beteiligten Agenten
	\item Den Aktionen der beteiligten Agenten
	\item Den erhaltenen Auszahlungen der Agenten
\end{itemize}
	
Konstruktoren:
\begin{itemize}\itemsep -10pt
\item \texttt{GameResult(Agent player1, Agent player2, int payOff1, int payOff2, boolean hasCooperated1, boolean hasCooperated2)}
\item[] Creates a new \texttt{GameResult}.
\item[] \texttt{player1}: the first player
\item[] \texttt{player2}: the second player
\item[] \texttt{payOff1}: the received payoff of the first player
\item[] \texttt{payOff2}: the received payoff of the second player
\item[] \texttt{hasCooperated1}: \texttt{true} if the first player cooperated, \texttt{false} otherwise
\item[] \texttt{hasCooperated2}: \texttt{true} if the second player cooperated, \texttt{false} otherwise
\end{itemize}

Methoden:
\begin{itemize}\itemsep -10pt
\item  \texttt{boolean hasAgent(Agent agent)}
\item[] Returns whether the given agent was one of the players of this game.
\item[] \texttt{agent}: the agent that should be checked
\item[] Returns: \texttt{true} if the given agent was one of the players of this game, \texttt{false} otherwise

\item \texttt{int getPayoff(Agent agent)}
\item[] Returns the received payoff of the given agent in this game, if he was one of the players; \(0\) otherwise
\item[] \texttt{agent}: the agent whose payoff shall be returned
\item[] Returns: the received payoff of the given agent in this game, if he was one of the players; \(0\) otherwise

\item \texttt{boolean hasCooperated(Agent agent)}
\item[] Returns \texttt{true} if the given agent was one of the players of this game and cooperated, \texttt{false} otherwise.
\item[] \texttt{agent}: the agent whose cooperation shall be returned
\item[] Returns: \texttt{true} if the given agent was one of the players of this game and cooperated, \texttt{false} otherwise

\item \texttt{Collection<Agent> getAgents()}
\item[] Returns the two players of this game.
\item[] Returns: the two players of this game
\end{itemize}

\subsubsection{Interface \texttt{Game}}
Extends: \texttt{Nameable}

Dieses Interface repräsentiert ein Stufenspiel.

Methoden:
\begin{itemize}\itemsep -10pt
\item[] \texttt{GameResult play(Agent player1, Agent player2, boolean player1Cooperates, boolean player2Cooperates)}
\item[] Lets two players with given cooperation decisions play this game against each other and returns the result of the game as \texttt{GameResult}.
\item[] \texttt{player1}: the first player
\item[] \texttt{player2}: the second player
\item[] \texttt{player1Cooperates}: \texttt{true} if the first player cooperates, \texttt{false} otherwise
\item[] \texttt{player2Cooperates}: \texttt{true} if the second player cooperates, \texttt{false} otherwise
\item[] Returns: the result of the game as \texttt{GameResult}
\end{itemize}

\subsubsection{Interface \texttt{Strategy}}
Extends: \texttt{Nameable}

Repräsentiert eine Strategie. Diese kann, muss aber nicht gemischt sein.

Methoden:
\begin{itemize}\itemsep -10pt
\item \texttt{boolean isCooperative(Agent player, Agent opponent, SimulationHistory history)}
\item[] Returns whether agent \texttt{player} would cooperate in a game against agent \texttt{opponent} using this strategy, given the \texttt{SimulationHistory} of the current adaption step. If the strategy is non-deterministic, the result may be random.
\item[] Returns: if agent \texttt{player} cooperates

\item \texttt{double getCooperationProbability(Agent player, Agent opponent, SimulationHistory history)}
\item[] Returns the probability with which agent \texttt{player} would cooperate in a game against agent \texttt{opponent} using this strategy, given the \texttt{SimulationHistory} of the current adaption step.
\item[] Returns: the probability with which agent \texttt{player} would cooperate in a game against agent \texttt{opponent} using this strategy, given the \texttt{SimulationHistory} of the current adaption step
\end{itemize}

\subsubsection{Class \texttt{PureStrategy}}
Implements: \texttt{Strategy}

Diese Klasse repräsentiert eine reine Strategie. Diese ist eindeutig bestimmt durch eine Bedingung an den gegnerischen Agenten basierend auf dem bisherigen Verlauf der Simulation, repräsentiert als \texttt{Predicate<Agent,SimulationHistory>}.

Konstruktoren:
\begin{itemize}\itemsep -10pt
\item \texttt{PureStrategy(Predicate<Agent,SimulationHistory> condition)}
\item[] Creates a new \texttt{PureStrategy}.
\item[] \texttt{condition}: the condition an opposing agent must meet in order for an agent using this strategy to cooperate in a game against him
\end{itemize}

\subsubsection{Class \texttt{CustomStrategy}}
//vorläufig

\subsubsection{Class \texttt{MixedStrategy}}
Implements: \texttt{Strategy}

Diese Klasse repräsentiert eine gemischte Strategie, die sich aus mehreren \texttt{PureStrategie}s zusammensetzt.

Konstruktoren://vorläufig
\begin{itemize}\itemsep -10pt
\item[] //wirft exception wenn listen nicht gleichlang oder wahrsch. ungültig (alle positiv und in summe 1)
\item \texttt{MixedStrategy(List<PureStrategies> pureStrategies, List<Integer> probabilites)}
\item[] Creates a new \texttt{MixedStrategy} consisting of the given \texttt{PureStrategy}s with the given probabilities.
\item[] \texttt{pureStrategies}: the \texttt{PureStrategy}s this strategy consists of
\item[] \texttt{probabilities}: the probabilities of the given \texttt{PureStrategy}s
\end{itemize}

\subsubsection{Class \texttt{Segment}}
//umbenennen!
Diese Klasse repräsentiert ein Segment. Dieses wird charakterisiert durch die Anzahl zugehöriger Agenten, die \texttt{id} der Gruppe, zu der das Segment gehört sowie initiale Kapital- und Strategieverteilung.

Konstruktoren:

Methoden:
\begin{itemize}\itemsep -10pt
\item \texttt{int getAgentCount()}
\item[] Returns the amount of agents in this segment.
\item[] Returns: the amount of agents in this segment

\item \texttt{int getGroupId()}
\item[] Returns the \texttt{id} of the group this segment is part of, if it is cohesive; \(-1\) otherwise.
\item[] Returns: the \texttt{id} of the group this segment is part of, if it is cohesive; \(-1\) otherwise

\item \texttt{DiscreteDistribution getCapitalDistribution()}
\item[] Returns the capital distribution of this segment
\item[] Returns: the capital distribution of this segment as \texttt{DiscreteDistribution}

\item \texttt{UniformFiniteDistribution<Strategy> getStrategyDistribution()}
\item[] Returns the strategy distribution of this segment
\item[] Returns: the strategy distribution of this segment as \texttt{UniformFiniteDistribution<Strategy>}
\end{itemize}

\subsubsection{Interface \texttt{PairBuilder}}
Dieses Interface repräsentiert einen Paarbildungsalgorithmus. Eine Implementierung nimmt eine Menge von Agenten sowie den bisherigen Verlauf des aktuellen Adaptionsschritts in Form eines \texttt{SimulationHistory}-Objektes entgegen und gibt ein Matching der Agenten zurück.

Methoden:
\begin{itemize}\itemsep -10pt
\item \texttt{Collection<AgentPair> buildPairs(Collection<Agent> agents, SimulationHistory history)}
\item[] Fasst die gegebene Menge von Agenten zu Paaren zusammen und gibt diese als \texttt{Collection} von \texttt{AgentPair}s zurück.
\item[] \texttt{agents}: die Agenten, aus denen Paare gebildet werden sollen
\item[] \texttt{history}: der bisherige Verlauf des Adaptionsschritts
\item[] Returns: die gebildeten Paare als \texttt{Collection<AgentPair>}
\end{itemize}

\subsubsection{Class \texttt{RandomPairBuilder}}
Implements: \texttt{PairBuilder}

Realisiert die zufällige Paarbildung von Agenten.

\subsubsection{Class \texttt{CooperationConsideringPairBuilder}}
Implements: \texttt{PairBuilder}

Realisiert die in der Spezifikation beschriebene \enquote{Paarbildung nach Wunsch}.

\subsubsection{Class \texttt{RandomCooperationConsideringPairBuilder}}
Implements: \texttt{PairBuilder}

Realisiert die in der Spezifikation beschriebene \enquote{Paarbildung nach Wunsch mit Zufall}.

Konstruktoren:
\begin{itemize}\itemsep -10pt
\item \texttt{RandomCooperationConsideringPairBuilder(double randomnessFactor)}
\item[] Erstellt einen neuen \texttt{RandomCooperationConsideringPairBuilder}.
\item[] \texttt{randomnessFactor}: der Zufallsfaktor für die Paarbildung; zwischen \(0\) und \(1\)
\end{itemize}

\subsubsection{Interface \texttt{SuccessQuantifier}}
Dieses Interface repräsentiert eine Erfolgsquantifizierung. Eine Implementierung nimmt eine Menge von Agenten sowie den bisherigen Verlauf des aktuellen Adaptionsschritts in Form eines \texttt{SimulationHistory}-Objektes entgegen und erstellt eine Rangliste der Agenten anhand deren Erfolg in den vergangenen Runden. Diese wird als Liste zurückgegeben.

Methoden:
\begin{itemize}\itemsep -10pt
\item \texttt{List<Agent> createRanking(Collection<Agent> agents, SimulationHistory history)}
\item[] Erstellt eine Rangliste der gegebenen Agenten und gibt diese als Liste zurück.
\item[] \texttt{agents}: die Agenten, aus denen eine Rangliste erstellt werden soll
\item[] \texttt{history}: der bisherige Verlauf des aktuellen Adaptionsschritts
\item[] Returns: die erstellte Rangliste als \texttt{List<Agent>}
\end{itemize}

\subsubsection{Class \texttt{TotalCapital}}
Implements: \texttt{SuccessQuantifier}

Realisiert die in der Spezifikation beschriebene \enquote{Absolutkapital}-Erfolgsquantifizierung

\subsubsection{Class \texttt{TotalPayoff}}
Implements: \texttt{SuccessQuantifier}

Realisiert die in der Spezifikation beschriebene \enquote{Absolutkapital ohne Initialkapital}-Erfolgsquantifizierung.

\subsubsection{Class \texttt{SlidingMean}}
Implements: \texttt{SuccessQuantifier}

Realisiert die in der Spezifikation beschriebene \enquote{Gleitender Durchschnitt}-Erfolgsquantifizierung.

Konstruktoren:
\begin{itemize}\itemsep -10pt
\item \texttt{SlidingMean(int windowSize)}
\item[] Erstellt ein neues \texttt{SlidingMean}-Objekt mit der gegebenen Fenstergröße.
\item[] \texttt{windowSize}: die Fenstergröße für die Durchschnittsberechnung
\end{itemize}

\subsubsection{Class \texttt{PayoffInLastAdapt}}
Implements: \texttt{SuccessQuantifier}

Realisiert die in der Spezifikation beschriebene \enquote{Auszahlung im letzten Adaptionsschritt}-Erfolgsquantifizierung.

\subsubsection{Interface \texttt{StrategyAdjuster}}
Dieses Interface repräsentiert einen Adaptionsmechanismus. Eine Implementierung nimmt eine nach Rang geordnete Liste von Agenten entgegen und passt die Strategien der Agenten an.

Methoden:
\begin{itemize}\itemsep -10pt
\item \texttt{void adaptStrategies(List<Agent> agents)}
\item[] Adaptiert die Strategien der gegebenen Agenten.
\item[] \texttt{agents}: die Agenten, deren Strategien adaptiert werden sollen, geordnet nach Erfolg im aktuellen Adaptionsschritt
\end{itemize}

\subsubsection{Class \texttt{ReplicatorDynamic}}
Implements: \texttt{StrategyAdjuster}

Realisiert den in der Spezifikation beschriebenen \enquote{Replicator Dynamic}-Anpassungsmechanismus.

Konstruktoren:
\begin{itemize}\itemsep -10pt
\item \texttt{ReplicatorDynamic(double alpha, double delta)}
\item[] Erstellt ein neues \texttt{ReplicatorDynamic}-Objekt mit den gegebenen Parametern \(\alpha\) und \(\delta\).
\item[] \texttt{alpha}: die Vergleichswahrscheinlichkeit
\item[] \texttt{delta}: die Anpassungswahrscheinlichkeit
\end{itemize}

\subsubsection{Class \texttt{PreferentialAdaption}}
Implements: \texttt{StrategyAdjuster}

Realisiert den in der Spezifikation beschriebenen \enquote{Preferential Adaption}-Anpassungsmechanismus.

Konstruktoren:
\begin{itemize}\itemsep -10pt
\item \texttt{PreferentialAdaption(double alpha, double delta)}
\item[] Erstellt ein neues \texttt{PreferentialAdaption}-Objekt mit den gegebenen Parameter \(\alpha\) und \(\delta\).
\item[] \texttt{alpha}: die Vergleichswahrscheinlichkeit
\item[] \texttt{delta}: die Anpassungswahrscheinlichkeit
\end{itemize}

\subsubsection{Interface \texttt{EquilibriumCriterion}}
Dieses Interface repräsentiert ein Gleichgewichtskriterium. Eine Implementierung nimmt am Ende eines Adaptionsschritts eine Liste der Agenten, geordnet nach Rang, sowie den Verlauf des Adaptionsschritts in Form eines \texttt{SimulationHistory}-Objektes entgegen und gibt zurück, ob sich ein Gleichgewicht eingestellt hat.

Methoden:
\begin{itemize}\itemsep -10pt
\item \texttt{boolean isEquilibrium(List<Agent> agents, SimulationHistory history)}
\item[] Gibt zurück, ob sich ein Gleichgewicht eingestellt hat.
\item[] \texttt{agents}: die Agenten des aktuellen Simulationslaufs
\item[] \texttt{history}: der Verlauf des aktuellen Adaptionsschritts
\item[] Returns: \texttt{true}, falls sich ein Gleichgewicht eingestellt hat, \texttt{false} sonst
\end{itemize}

\subsubsection{Abstract Class \texttt{CountingEquilibriumCriterion}}
Implements: \texttt{EquilibriumCriterion}

Diese Klasse repräsentiert ein Gleichgewichtskriterium, dass jede Runde eine bestimmte Bedingung prüft und ein Gleichgewicht erkennt, wenn diese Bedingung für eine bestimmte Zahl aufeinanderfolgender Adaptionsschritte erfüllt ist. Diese Bedingung und die nötige Zahl aufeinanderfolgender Adaptionsschritte sind als Schablonenmethoden implementiert.

Methoden:
\begin{itemize}\itemsep -10pt
\item \texttt{abstract boolean hasEquilibriumCondition(List<Agent> agents, SimulationHistory history)}
\item[] Gibt zurück, ob die Gleichgewichtsbedingung für diesen Adaptionsschritt erfüllt ist.
\item[] \texttt{agents}: die Agenten des aktuellen Simulationslaufs
\item[] \texttt{history}: der Verlauf des aktuellen Adaptionsschritts
\item[] Returns: \texttt{true}, falls die Bedingung erfüllt ist, \texttt{false} sonst

\item \texttt{abstract boolean longEnough(int steps)}
\item[] Gibt zurück, ob \texttt{steps} größer oder gleich der nötigen Zahl aufeinanderfolgender Gleichgewichtsbedingung-erfüllender Adaptionsschritte ist.
\item[] \texttt{steps}: die Anzahl von Adaptionsschritten, für die geprüft werden soll, ob sie in obigem Sinne hinreichend groß ist
\item[] Returns: \texttt{true}, falls \texttt{steps} groß genug ist, \texttt{false} sonst
\end{itemize}

\subsubsection{Class \texttt{StrategyEquilibrium}}
Implements: \texttt{CountingEquilibriumCriterion}

Realisiert das in der Spezifikation beschriebene \enquote{Strategie-Gleichgewichtskriterium}.

Konstruktoren:
\begin{itemize}\itemsep -10pt
\item \texttt{StrategyEquilibrium(double alpha, int G)}
\item[] Erzeugt ein neues \texttt{StrategyEquilibrium}-Objekt mit den gegebenen Parametern.
\item[] \texttt{alpha}: die Strenge des Gleichgewichts
\item[] \texttt{G}: die minimale Zahl aufeinanderfolgender Gleichgewichtsbedingung-erfüllender Adaptionsschritte
\end{itemize}

\subsubsection{Class \texttt{RankingEquilibrium}}
Implements: \texttt{CountingEquilibriumCriterion}

Realisiert das in der Spezifikation beschriebene \enquote{Rang-Gleichgewichtskriterium}.

Konstruktoren:
\begin{itemize}\itemsep -10pt
\item \texttt{RankingEquilibrium(double alpha, int G)}
\item[] Erzeugt ein neues \texttt{RankingEquilibrium}-Objekt mit den gegebenen Parametern.
\item[] \texttt{alpha}: die Strenge des Gleichgewichts
\item[] \texttt{G}: die minimale Zahl aufeinanderfolgender Gleichgewichtsbedingung-erfüllender Adaptionsschritte
\end{itemize}

\subsubsection{Interface \texttt{Distribution<E>}}
Dieses Interface repräsentiert eine Wahrscheinlichkeitsverteilung über einem Träger von Objekten des Typs \texttt{E}. Eine Implementierung gibt auf Anfrage die Wahrscheinlichkeit eines bestimmten Objekts zurück und stellt einen \texttt{Picker<E>} zur Verfügung, mit dem zufällig Objekte aus der Wahrscheinlichkeitsverteilung gezogen werden können.

Methoden
\begin{itemize}\itemsep -10pt
\item \texttt{double getProbability(E object)}
\item[] Gibt die Wahrscheinlichkeit des gegebenen Objekts in dieser Verteilung zurück.
\item[] \texttt{element}: das Objekt, dessen Wahrscheinlichkeit abgefragt werden soll
\item[] Returns: die Wahrscheinlichkeit des gegebenen Objekts in dieser Verteilung

\item \texttt{Picker<E> getPicker()}
\item[] Gibt einen \texttt{Picker<E>} für diese Wahrscheinlichkeitsverteilung zurück.
\item[] Returns: einen \texttt{Picker<E>} für diese Wahrscheinlichkeitsverteilung
\end{itemize}

\subsubsection{Interface \texttt{Picker<E>}}
Ein \texttt{Picker<E>} wird von Implementierungen des \texttt{Distribution<E>}-Interfaces zur Verfügung gestellt, um Objekte aus der entsprechenden Wahrscheinlichkeitsverteilung zu ziehen.

Methoden:
\begin{itemize}\itemsep -10pt
\item \texttt{E pickOne()}
\item[] Zieht ein Objekt aus der Wahrscheinlichkeitsverteilung und gibt dieses zurück.
\item[] Returns: das aus der Wahrscheinlichkeitsverteilung gezogene Objekt

\item \texttt{Collection<E> pickMany(int i)}
\item[] Zieht mehrere Objekte aus der Wahrscheinlichkeitsverteilung und gibt diese als \texttt{Collection<E>} zurück.
\item[] \texttt{i}: die Anzahl der Objekte, die aus der Wahrscheinlichkeitsverteilung gezogen werden sollen
\item[] Returns: die aus der Wahrscheinlichkeitsverteilung gezogenen Objekte als \texttt{Collection<E>}
\end{itemize}

\subsubsection{Interface \texttt{DiscreteDistribution}}
Implements: \texttt{Distribution<Integer>}

Repräsentiert eine diskrete Wahrscheinlichkeitsverteilung auf den ganzen Zahlen. Eine Implementierung berechnet für einen gegebenen Wert \(q \in (0,1)\) ein Intervall \(I_q = [a,b]\) (\(a,b \in \mathbb{Z}, a < b\)), sodass ein zufällig gezogener Wert aus der Verteilung mindestens mit der Wahrscheinlichkeit \(q\) in \(I_q\) liegt. Die Grenzen dieses Intervalls können abgefragt werden.

Methoden:
\begin{itemize}\itemsep -10pt
\item \texttt{int getSupportMin(double q)}
\item[] Gibt die linke Grenze des Intervalls \(I_q\) zurück.
\item[] \texttt{q}: der Wert \(q\) zur Berechnung des Intervalls \(I_q\)
\item[] Returns: die linke Grenze des Intervalls \(I_q\)

\item \texttt{int getSupportMax(double q)}
\item[] Gibt die rechte Grenze des Intervalls \(I_q\) zurück.
\item[] \texttt{q}: der Wert \(q\) zur Berechnung des Intervalls \(I_q\)
\item[] Returns: die rechte Grenze des Intervalls \(I_q\)
\end{itemize}

\subsubsection{Class \texttt{PoissonDistribution}}
Implements: \texttt{DiscreteDistribution}

Repräsentiert eine Poisson-Verteilung.

Konstruktoren:
\begin{itemize}\itemsep -10pt
\item \texttt{PoissonDistribution(double lambda)}
\item[] Erzeugt ein neues \texttt{PoissonDistribution}-Objekt mit dem Parameter \texttt{lambda}.
\item[] \texttt{lambda}: der variable Parameter in der Poissonverteilung
\end{itemize}

\subsubsection{Class \texttt{BinomialDistribution}}
Implements: \texttt{DiscreteDistribution}

Repräsentiert eine Bionomialverteilung.

Konstruktoren:
\begin{itemize}\itemsep -10pt
\item \texttt{BinomialDistribution(int a, int b, double p)}
\item[] Erzeugt ein neues \texttt{BinomialDistribution}-Objekt mit den gegebenen Parametern.
\item[] \texttt{a}: die linke Grenze der Verteilung
\item[] \texttt{b}: die rechte Grenze der Verteilung
\item[] \texttt{p}: der Wahrscheinlichkeitsparameter \(p \in [0,1]\) der Binomialverteilung
\end{itemize}

\subsubsection{Class \texttt{DiscreteUniformDistribution}}
Implements: \texttt{DiscreteDistribution}

Repräsentiert eine diskrete Gleichverteilung.

Konstruktoren:
\begin{itemize}\itemsep -10pt
\item \texttt{DiscreteUniformDistribution(int a, int b)}
\item[] Erzeugt ein neues \texttt{DiscreteUniformDistribution}-Objekt mit den gegebenen Parametern.
\item[] \texttt{a}: die linke Grenze der Verteilung
\item[] \texttt{b}: die rechte Grenze der Verteilung
\end{itemize}

\subsubsection{Interface \texttt{FiniteDistribution<E>}}
Extends: \texttt{Distribution<E>}

Repräsentiert eine Wahrscheinlichkeitsverteilung über Objekten des Typs \texttt{E} mit endlichem Träger. Implementierungen geben auf Abfrage den Träger als \texttt{Collection<E>} zurück.

Methoden:
\begin{itemize}\itemsep -10pt
\item \texttt{Collection<E> getSupport()}
\item[] Gibt den Träger dieser Wahrscheinlichkeitsverteilung zurück.
\item[] Returns: den Träger dieser Wahrscheinlichkeitsverteilung als \texttt{Collection<E>}
\end{itemize}

\subsubsection{Class \texttt{UniformFiniteDistribution<E>}}
Implements: \texttt{FiniteDistribution<E>}

Repräsentiert eine Gleichverteilung über Objekten des Typs \texttt{E} mit endlichem Träger. Implementierungen bieten Methoden zum Hinzufügen und Entfernen von Objekten.

Konstruktoren:
\begin{itemize}\itemsep -10pt
\item \texttt{UniformFiniteDistribution<E>()}
\item[] Erzeugt ein neues \texttt{UniformFiniteDistribution<E>}-Objekt mit leerem Träger.

\item \texttt{UniformFiniteDistribution<E>(Collection<E> ojects)}
\item[] Erzeugt ein neues \texttt{UniformFiniteDistribution<E>}-Objekt mit den gegebenen Objekten als Träger.
\item[] \texttt{objects}: die Objekte des Trägers dieser Wahrscheinlichkeitsverteilung
\end{itemize}

Methoden:
\begin{itemize}\itemsep -10pt
\item \texttt{void addObject(E object)}
\item[] Fügt das gegebene Objekt zum Träger hinzu.
\item[] \texttt{object}: das Objekt, das dem Träger hinzugefügt werden soll

\item \texttt{void addObjects(Collection<E> objects)}
\item[] Fügt die gegebenen Elemente zum Träger hinzu.
\item[] \texttt{objects}: die Objekte, die dem Träger hinzugefügt werden sollen

\item \texttt{boolean deleteObject(E object)}
\item[] Entfernt das gegebene Objekt aus dem Träger, falls enthalten.
\item[] \texttt{object}: das Objekt, das aus dem Träger entfernt werden soll
\item[] Returns: \texttt{true}, falls das gegebene Objekt im Träger enthalten war und entfernt wurde, andernfalls \texttt{false}
\end{itemize}

\subsection{Paket \texttt{edu.kit.loop.Controller}}
Dieses Paket enthält die Kontrollstruktur des Programms und dient als Verbindung zwischen dem View und dem Model.
Die einzelnen Kontroller sind in einer baumartigen Struktur angeordnet und die Wurzel ist der \enquote{Headcontroller}. Dieser dient als Schnittstelle zwischen dem Model und dem Rest der Kontrollstruktur.

\subsubsection{Class \texttt{HeadController}}
Diese Klasse repräsentiert die Schnittstelle zwischen dem Model und dem Rest der Kontrollstruktur. Sie kann dem Simulator eine neue Konfiguration in Auftrag geben, Populationen zum abspeichern übergeben oder eine abgespeicherte Konfiguration erneut öffnen.

Konstruktoren:
\begin{itemize}\itemsep -10pt
\item \texttt{CentralController(Simulator simulator, ConfigurationParameterRepository repository)}
\item[] Erstellt einen neuen HeadController, sichert die Referenz auf den Simulator und lädt das übergebene Repository.
\item[] \texttt{simulator}: der Simulator der die Simulationen mit gegebener Konfiguration startet
\item[] \texttt{repository}: Repository, das die verschiedenen Strategien, Paarungsalgorithmen und die weiteren verfügbaren Einstellungsmöglichkeiten enthält
\end{itemize}

Methoden:
\begin{itemize}\itemsep -10pt
\item \texttt{void setConfiguration(Configuration)}
\item[] Stellt eine Configuration im Kontroller ein


\item \texttt{Configuration getConfiguration()}
\item[] Gibt die aktuell eingestellte Konfiguration zurück
\item[] Returns: die aktuell eingestellte Konfiguration

\item \texttt{void addSimulationToHistory(Simulation)}
\item[] Fügt eine neue Simulation in die Simulationshistorie hinzu

\item \texttt{void saveStrategy(Strategy strategy)}
\item[] Speichert die ausgewählte Strategie ab.

\item\texttt{void saveGame(Game stufenspiel)}
\item[] Speichert das ausgewählte Stufenspiel ab

\item \texttt{void saveConfiguration(Configuration configuration)}
\item[] Speichert die ausgewählte Konfiguration an einem gewünschten Ort ab

\item \texttt{void savePopulation(Population population)}
\item[] Speichert die ausgewählte Population ab

\item \texttt{void saveGroup(Group group)}
\item[] Speichert die ausgewählte Gruppe ab

\item \texttt{void updateSimulation(Simulation simulation)}
\item[] Updatet den Fortschritt einer gewünschten Simulation

\item \texttt{void runConfiguration(Configuration configuration}
\item[] Gibt dem Simulator den Befehl eine neue Simulation mit der ausgewählten Konfiguration zu starten
\end{itemize}

\subsubsection{Class \texttt{MainController}}
Diese Klasse repräsentiert den Kontroller der für das Hauptfenster zuständig ist.

Konstruktoren:
\begin{itemize}\itemsep -10pt
\item \texttt{MainController()}
\item[] Erstellt einen neuen MainController
\end{itemize}

Methoden:
\begin{itemize}\itemsep -10pt
\item \texttt{void setConfiguration(Configuration)}
\item[] Stellt eine Configuration im Kontroller ein


\item \texttt{void configure()}
\item[] Öffnet das Konfigurationsfenster

\item \texttt{void addSimulationToHistory(Simulation)}
\item[] Fügt eine neue Simulation in die Simulationshistorie hinzu

\item \texttt{void updateSimulation(Simulation simulation)}
\item[] Updatet den Fortschritt einer gewünschten Simulation
\end{itemize}

\subsubsection{Class \texttt{BasicsController}}
Diese Klasse repräsentiert den Kontroller der für die Basisansicht im Hauptfenster zuständig ist.

Methoden:
\begin{itemize}\itemsep -10pt
\item \texttt{void setConfiguration(Configuration)}
\item[] Stellt eine Configuration im Kontroller ein


\item \texttt{void configure()}
\item[] Öffnet das Konfigurationsfenster
\end{itemize}

\subsubsection{Class \texttt{GraphicController}}
Diese Klasse repräsentiert den Kontroller der für die Ergebnisausgabe im Hauptfenster zuständig ist.

Methoden:
\begin{itemize}\itemsep -10pt
\item \texttt{void showSimulation(Simulation simulation)}
\item[] Zeigt die Ergebnisse für die ausgewählte Simulation an

\item \texttt{void repaint()}
\item[] Zeichnet die Graphen und Diagramme auf der ausgewählten Seite mit den neuen Werten


\item \texttt{void updateValues()}
\item[] Updatet alle im Fenster verstellbaren Parameter

\item \texttt{void setPageNumber(int page)}
\item[] Stellt die Seitenzahl auf der der Benutzer sich gerade befindet ein
\end{itemize}


\subsubsection{Class \texttt{HistoryController}}
Diese Klasse repräsentiert den Kontroller der für die Simulationshistorie zuständig ist.

Konstruktoren:
\begin{itemize}\itemsep -10pt
\item \texttt{HistoryController()}
\item[] Erstellt einen neuen HistoryController
\end{itemize}

Methoden:
\begin{itemize}\itemsep -10pt
\item \texttt{void addSimulationToHistory(Simulation)}
\item[] Fügt eine neue Simulation in die Simulationshistorie hinzu

\item \texttt{void updateSimulationProgress(Simulation simulation)}
\item[] Updatet den Fortschritt einer gewünschten Simulation
\end{itemize}

\subsubsection{Class \texttt{MenuController}}
Diese Klasse repräsentiert den Kontroller der für das Menu zuständig ist.

Konstruktoren:
\begin{itemize}\itemsep -10pt
\item \texttt{MenuController(Repository<t> repository)}
\item[] Erstellt einen neuen MenuController und dieser erhält das Repository mit allen verfügbaren Plugins
\end{itemize}

Methoden:
\begin{itemize}\itemsep -10pt
\item \texttt{void saveconfiguration(Configuration configuration)}
\item[] Speichert die gewünschte Konfiguration ab

\item \texttt{void loadConfiguration()}
\item[] Öffnet ein Dialogfenster in dem der Benutzer eine abgespeicherte Konfiguration zum öffnen auswählen kann

\item \texttt{void addGame()}
\item[] Öffnet das \enquote{Neues Stufenspiel erstellen}-Fenster

\item \texttt{void addStrategy()}
\item[] Öffnet das \enquote{Neue Strategy erstellen}-Fenster

\item \texttt{void addPopulation()}
\item[] Öffnet das \enquote{Neue Population erstellen}-Fenster

\item \texttt{void addGroup()}
\item[] Öffnet das \enquote{Neue Gruppe erstellen}-Fenster

\item \texttt{void close()}
\item[] Schließt das Fenster

\item \texttt{void runConfiguration(Configuration configuration}
\item[] Gibt dem HeadController den Befehl eine neue Simulation mit der ausgewählten Konfiguration zu starten

\item \texttt{void help()}
\item[] Zeigt die Hilfe für den Benutzer an
\end{itemize}

\subsubsection{Class \texttt{GroupController}}
Diese Klasse repräsentiert den Kontroller der für das \enquote{Neue Gruppe erstellen}-Fenster zuständig ist.

Konstruktoren:
\begin{itemize}\itemsep -10pt
\item \texttt{Groupcontroller(Repository<t> repository)}
\item[] Erstellt einen neuen GroupController und dieser erhält das Repository mit allen verfügbaren Plugins
\end{itemize}

Methoden:
\begin{itemize}\itemsep -10pt
\item \texttt{void addSegment(}
\item[] Fügt der Gruppe ein neues Segment hinzu

\item \texttt{void deleteSegment(String segmentId)}
\item[] Löscht eine über einen String klar definierbares Segment aus der Gruppe

\item \texttt{void setSegmentSize(String segmentId, int size)}
\item[] Setzt die Grösse des ausgewählten Segments auf die angegebene Anzahl an Agenten

\item \texttt{int getSegmentSize(String segmentId)}
\item[] Gibt die Grösse des ausgewählten Segments zurück
\item[] Returns: die Anzahl an Agenten des ausgewählten Segments

\item \texttt{String getGroupName()}
\item[] Gibt den Namen der aktuellen Gruppe zurück
\item[] Returns: der Name der aktuell ausgewählten Gruppe

\item \texttt{void setGroupName(String name)}
\item[] Setzt den Namen der Gruppe auf den vom Nutzer eingegebenen Namen

\item \texttt{String getGroupDescription()}
\item[] Gibt die Beschreibung der aktuellen Gruppe zurück
\item[] Returns: die Beschreibung der aktuell ausgewählten Gruppe

\item \texttt{void setGroupDescription(String description)}
\item[] Setzt die Beschreibung der Gruppe auf die vom Nutzer eingegebenen Beschreibung

\end{itemize}

\subsubsection{Class \texttt{SegmentController}}
Diese Klasse repräsentiert den Kontroller der das Tab mit den Segmenteinstellungen verwaltet

Konstruktoren:
\begin{itemize}\itemsep -10pt
\item \texttt{Segmentcontroller(Repository<t> repository)}
\item[] Erstellt einen neuen SegmentController und dieser erhält das Repository mit allen verfügbaren Plugins
\end{itemize}

Methoden:
\begin{itemize}\itemsep -10pt
\item \texttt{void addStrategy(String strategyId}
\item[] Fügt die ausgewählte Strategy in die Liste der möglichen Strategien

\item \texttt{void removeStrategy(String segmentId)}
\item[] Entfernt sie ausgewählte Strategy aus der Liste der möglichen Startegien

\item \texttt{void selectCapitalDistribution(String segmentId, String distributionId)}
\item[] Setzt die Kapitalverteilungsfunktion des ausgewählten Segments aufs die ausgewählte Funktion

\item \texttt{int getCapitalDistribution(String segmentId)}
\item[] Gibt die Kapitalverteilung des ausgewählten Segments zurück
\item[] Returns: die Kapitalverteilungsfunktion des ausgewählten Segments

\item \texttt{getDistributionParameters(segmentId) : ???}
\item[] Gibt die Kapitalverteilungsparameter des ausgewählten Segments zurück
\item[] Returns: die Kapitalverteilungsparameter des ausgewählten Segments

\item \texttt{setDistributionParameters(String segmentId)}
\item[] Setzt die Verteilungsfunktionparameter auf die ausgewählten Werte


\end{itemize}


\subsubsection{Class \texttt{PopulationController}}
Diese Klasse repräsentiert den Kontroller der für das \enquote{Neue Population erstellen}-Fenster zuständig ist.

Konstruktoren:
\begin{itemize}\itemsep -10pt
\item \texttt{PopulationController(Repository<t> repository)}
\item[] Erstellt einen neuen PopulationController und dieser erhält das Repository mit allen verfügbaren Gruppen
\end{itemize}

Methoden:
\begin{itemize}\itemsep -10pt
\item \texttt{void addGroup(String groupId)}
\item[] Fügt eine über einen String klar definierbare Gruppe aus dem Repository zur Population hinzu

\item \texttt{void deleteGroup(String groupId)}
\item[] Löscht eine über einen String klar definierbare Gruppe aus der Population

\item \texttt{void setGroupSize(String groupId, int size)}
\item[] Setzt die Grösse der ausgewählten Gruppe auf die angegebene Anzahl an Agenten

\item \texttt{int getGroupSize(String groupId)}
\item[] Gibt die Grösse der ausgewählten Gruppe zurück
\item[] Returns: die Anzahl an Agenten der ausgewählten Gruppe

\item \texttt{String getPopulationName()}
\item[] Gibt den Namen der aktuellen Population zurück
\item[] Returns: der Name der aktuell ausgewählten Population

\item \texttt{void setPopulationName(String name)}
\item[] Setzt den Namen der Population auf den vom Nutzer eingegebenen Namen

\item \texttt{String getPopulationDescription()}
\item[] Gibt die Beschreibung der aktuellen Population zurück
\item[] Returns: die Beschreibung der aktuell ausgewählten Population

\item \texttt{void setPopulationDescription(String description)}
\item[] Setzt die Beschreibung der Population auf die vom Nutzer eingegebenen Beschreibung

\end{itemize}


\subsubsection{Class \texttt{NewGameController}}
Diese Klasse repräsentiert den Kontroller der für das \enquote{Neues Stufenspiel erstellen}-Fenster zuständig ist.

Konstruktoren:
\begin{itemize}\itemsep -10pt
\item \texttt{NewGameController()}
\item[] Erstellt einen neuen NewGameController
\end{itemize}

Methoden:
\begin{itemize}\itemsep -10pt
\item \texttt{void setName()}
\item[] Stellt den Namen des Stufenspiels auf den im Fenster eingegebenen Namen


\item \texttt{Game getGame()}
\item[] Gibt das aktuell eingestellte Stufenspiel zurück
\item[] Returns: das aktuell eingestellte Stufenspiel

\item \texttt{void setDescription()}
\item[] Stellt die Beschreibung des Stufenspiels auf die im Fenster eingegebene Beschreibung

\item \texttt{void setPayments()}
\item[] Stellt die Werte der Matrix des Stufenspiels auf die im Fenster eingegebenen Werte

\item\texttt{void save()}
\item[] Speichert das aktuell eingegebene Stufenspiel ab

\item \texttt{void reset()}
\item[] Leert die Felder im Fenster für Name, Beschreibung und Auszahlungen

\end{itemize}

\subsection{Paket \texttt{edu.kit.loop.model.plugin}}
//Beschreibung!

\subsubsection{Class \texttt{PluginLoader}}

Diese Klasse stellt die Funktionalität zum Laden von Plugins mittels der Java \texttt{ServiceLoader}-API zur Verfügung

Methoden:

\begin{itemize}\itemsep -10pt
	\item \underline{\texttt{static <T> List<T> loadPlugins()}}
	\item[] Lädt alle verfügbaren Plugins des angegebenen Typs \texttt{T}
	\item[] \texttt{<T>} Plugintyp der geladen werden soll
	\item[]Returns: eine Liste der geladenen Plugins
\end{itemize}

\subsubsection{abstract Class \texttt{PluginControl extends javafx.scene.layout.Pane}}

Diese Klasse dient als Steuerelement für die Konfiguration eines Plugins. Sie stellt eine Funktionalität zum Abfragen der vom Benutzer eingegebenen Konfigurationsparameter zur Verfügung.

Methoden:

\begin{itemize}\itemsep -10pt
	\item \texttt{abstract List<double> getParameters()}
	\item[] Gibt eine Liste der eingegebene Parameter für die Pluginkonfiguration zurück
	\item[] Returns: eine Liste der eingegebenen \texttt{double}-Parameter
\end{itemize}

\subsubsection{Class \texttt{TextFieldPluginControl extends PluginControl}}

Diese Klasse ist eine Implementierung eines  \texttt{PluginControl}, bei der die Eingabe der Konfigurationsparameter über Textfelder erfolgt. Die Eingabe wird weiterhin bereits auf Korrektheit überprüft.

Konstruktoren:

\begin{itemize}\itemsep -10pt
	\item \texttt{TextFieldPluginControl(List<Parameter> params)}
	\item[] Erzeugt eine Instanz, die die übergebenen \texttt{Parameter} konfigurierbar macht.
	\item[] \texttt{params}: eine Liste der konfigurierbaren Parameter
\end{itemize}

Methoden:

\begin{itemize}\itemsep -10pt
	\item \texttt{void addParameter(Parameter param)}
	\item[] Fügt dem Steuerelement einen \texttt{Parameter} hinzu, der dann über das Steuerelement konfigurierbar ist
	\item[] \texttt{param}: Der Parameter, der zum Steuerelement hinzugefügt werden soll.
	\item \texttt{void addParameters(List<Parameter> params)}
	\item[] Fügt dem Steuerelement eine Liste von \texttt{Parameter}n hinzu, die dann über das Steuerelement konfigurierbar sind
	\item[] \texttt{params}: Die Parameter, die zum Steuerelement hinzugefügt werden sollen.
\end{itemize}

\subsubsection{Interface \texttt{PluginRenderer}}

Dieses Interface stellt die Funktionalität zum Erzeugen eines Konfigurations-Steuerelemet für Plugins in einer JavaFX Anwendung bereit.

Methoden:

\begin{itemize}\itemsep -10pt
	\item \texttt{PluginControl render()}
	\item[] Liefert eine \texttt{PluginControl}-Instanz zurück, die in ein JavaFX-Fenster eingefügt werden kann
\end{itemize}

\subsubsection{Class \texttt{GenericRenderer implements PluginRenderer}}

Diese Klasse implementiert das \texttt{PluginRenderer}-Interface und erzeugt für ein beliebiges Plugin ein generisches, Textfeld basiertes Konfigurations-Steuerelement (\texttt{TextFieldPluginControl}-Instanz).

Konstruktoren:

\begin{itemize}\itemsep -10pt
	\item \texttt{GenericRenderer(plugin: Plugin)}
	\item[] Erzeugt eine Instanz, die für das gegebene Plugin ein Konfigurations-Steuerelement erstellt
	\item[] \texttt{plugin}: Das Plugin welches der Renderer darstellen soll
	\item \texttt{GenericRednerer(params: List<Parameter>)}
	\item[] Erzeugt eine Instanz, die ein Steuerelement mit den gegebenen \texttt{Parameter} erzeugt.
	\item[] \texttt{params}: Eine Liste der konfigurierbaren Parameter eines Plugins
\end{itemize} 

\subsubsection{Class \texttt{Parameter implements Nameable}}

Diese Klasse definiert einen Konfigurationsparameter für Plugins. Sie bietet Funktionaliäten zum Festlegen des Wertebereichs und der Granularität des Konfigurationsparameter.

Konstruktoren:
\begin{itemize}\itemsep -10pt
	\item \texttt{Parameter()}
	\item[] Erzeugt eine neue Instanz mit unbeschränktem Wertebereich 
	\item \texttt{Parameter(double minVal)}
	\item[] Erzeugt eine neue Instanz mit nach unten beschränktem Wertebereich
	\item[] \texttt{minVal}: untere Schranke des Wertebereichs
	\item \texttt{Parameter(double maxVal)}
	\item[] Erzeugt eine neue Instanz mit nach oben beschränktem Wertebereich
	\item[] \texttt{maxVal}: obere Schranke des Wertebereichs
	\item \texttt{Parameter(double minVal, double maxVal)}
	\item[] Erzeugt eine neue Instanz mit nach unten und oben beschränktem Wertebereich
	\item[] \texttt{minVal}: untere Schranke des Wertebereichs
	\item[] \texttt{maxVal}: obere Schranke des Wertebereichs
	\item \texttt{Parameter(double minVal, double maxVal, double stepSize)}
	\item[] Erzeugt eine neue Instanz mit nach oben und unten beschränktem Wertebereich, sowie eingeschränkter Granularität
	\item[] \texttt{minVal}: untere Schranke des Wertebereichs
	\item[] \texttt{maxVal}: obere Schranke des Wertebereichs
	\item[] \texttt{stepSize}: Granularität des Parameters
	
\end{itemize}

Methoden:

\begin{itemize}\itemsep -10pt
	\item \texttt{double getMinValue()}
	\item[] Gibt die untere Schranke des gültigen Wertebereichs zurück
	\item[] Returns: die untere Schranke des gültigen Wertebereichs
	\item \texttt{double getMaxValue()}
	\item[] Gibt die obere Schranke des gültigen Wertebereichs zurück
	\item[]Returns: die obere Schranke des gültigen Wertebereichs
	\item \texttt{double getStepSize()}
	\item[] Gibt die Granularität des Wertebereichs zurück
	\item[]Returns: die Granularität des Wertebereichs
\end{itemize}

\subsubsection{Class \texttt{ParameterValidator}}

Diese Klasse bietet Funktionalitäten zum Validieren einer Parameterbelegung in Form von statischen Hilfsmethoden.

Methoden:

\begin{itemize} \itemsep -10pt
	\item \underline{\texttt{static boolean isValueValid(double val, Parameter param)}}
	\item[] Prüft, ob der gegeben Wert im Wertebereich des gegebenen Parameters liegt
	\item[] \texttt{val}: der Wert der überprüft werden soll
	\item[] \texttt{param}: der Parameter, dessen Wertebereich für die Überprüfung genutzt werden soll.
	\item[] Returns: \texttt{true}, wenn der Wert im Wertebereich des Parameters liegt \texttt{false} andernfalls
	\item \underline{\texttt{static double getClosestValid(double val, Parameter param)}}
	\item[] Gibt den Wert im Wertebereich des Parameters zurück, der am nächsten an dem gegebenen Wert liegt
	\item[] \texttt{val}: der Wert, der beim ermitteln betrachtet wird
	\item[] \texttt{param}: der Parameter, desssen Wertebereich betrachtet wird
	\item[] Returns: der Wert aus dem Wertebereich des Parameters, der dem gegebenen Wert am nächsten liegt
\end{itemize}

\subsubsection{abstract Class \texttt{Plugin<T> implements  Nameable}}

Diese Klasse stellt ein dynamisch ladbares Plugin da. Sie ist ein generischer Container für eine Klasse, die die eigentliche Pluginfunktionalität bereitstellt. Sie bietet Funktionalitäten zum erzeugen neuer parametrisierter Instanzen der Funktionalitätsklasse. Eine Liste der geforderten Konfigurationsparameter kann abgefragt werden.

Parameter:
\begin{itemize}\itemsep -10pt
	\item \texttt{<T>} Der Typ der Klasse, welche die Pluginfunktionalität bereit stellt. 
\end{itemize}

Methoden:
\begin{itemize}\itemsep -10pt
	\item \texttt{PluginRenderer getRenderer()}
	\item[] Gibt eine \texttt{PluginRenderer}-Instanz zum Erzeugen eines Konfigurations-Steuerelements für das Plugin zurück
	\item[] Returns: eine Instanz eines \texttt{PluginRenderer}s
	\item \texttt{abstract List<Parameter> getParameters()}
	\item[] Gibt eine Liste der Konfigurationsparameter des Plugins zurück
	\item[] Returns: eine Liste der Konfigurationsparameter
	\item \texttt{abstract T getNewInstance(List<double> params)}
	\item[] Erzeugt eine neue Instanz der Klasse, die die Pluginfunktionalität bereitstellt und parametrisiert sie mit den übergebenen Werten der Konfigurationsparametern
	\item[] \texttt{params}: eine Liste mit den Werten der Konfigurationsparameter
	\item[] Returns: eine neue Instanz der Klasse, die die Pluginfunktionalität zur Verfügung stellt
\end{itemize}

\subsection{Paket \texttt{edu.kit.loop.model.repository}}
Dieses Paket beinhaltet die Funktionalität der zentralen Repositorys, über die alle aktuell im System hinterlegten Spiele, Strategien, Populationen, Paarbildungsalgorithmen, Erfolgsquantifizierungen, Adaptionsmechanismen und Gleichgewichtskriterien abgefragt werden können.


\subsubsection{Inteface \texttt{Repository<T>}}

Ein Repository speichert eine Zuordnung von Namen und Entitäten und stellt diese auf Anfrage zur Verfügung.

Parameter:
\begin{itemize}\itemsep -10pt
	\item \texttt{<T>} Der Typ der Entitäten, die in dem Depot gespeichert werden sollen
\end{itemize}


Methoden:
\begin{itemize}\itemsep -10pt
	\item \texttt{T getEntityByName(String name)}
	\item[] Gibt die zu dem Namen gehörige Entität zurück oder \texttt{null} falls das Depot keine Entität unter dem gegebene Namen speichert
	\item[] \texttt{name}: der Name der Entität die abgefragt werden soll
	\item[] Returns: die zu dem Namen gehörige Entität oder \texttt{null}
	
	\item \texttt{boolean addEntity(String name, T entity)}
	\item[] Fügt dem Depot eine neue Entität unter dem gegebene Namen hinzu
	\item[] \texttt{name}: der Name der Entität, die hinzugefügt werden soll
	\item[] \texttt{entity}: Die Entität, die hinzugefügt werden soll
	\item[] Returns: \texttt{true}, wenn die Entität erfolgreich hinzugefügt wurde und \texttt{false}, wenn bereits eine Entität unter dem Name im Depot existiert 
	
	\item \texttt{boolean containsEntityName(String name)}
	\item[] Gibt zurück, ob das Depot eine Entität unter dem gegebenen Namen enthält
	\item[] \texttt{name}: der Name der geprüft werden soll
	\item[] Returns: \texttt{true}, wenn das Depot eine Entität unter diesem Namen speichert und \texttt{false}, wenn nicht
	
\end{itemize}

\subsubsection{Class \texttt{HashMapRepository implements Repository<T>}}

// Im falschen Paket

Ein Implementierung des \texttt{Repository<T>}-Interface, die Entitäten in einer \texttt{HashMap} speichert.

Parameter:

\begin{itemize}\itemsep -10pt
	\item \texttt{<T>} Der Typ der Entitäten, die in dem Depot gespeichert werden sollen
\end{itemize}

Konstruktoren:

\begin{itemize} \itemsep -10pt
	\item \texttt{HasMapRepository()}
	\item[] Erzeugt eine neue Instanz des Depots
\end{itemize}


\subsubsection{Class \texttt{FileIO}}

Diese Klasse stellt Funktionalitäten zum Laden und Speichern von Entitäten in Dateien zur Verfügung.

Methoden:

\begin{itemize}\itemsep -10pt
	\item \underline{\texttt{static <T> T loadEntity(File file)}}
	\item[] Lädt die Entität des Typs \texttt{T} aus der angegebenen Datei 
	\item[] \texttt{<T>} Entitätentyp der geladen werden soll
	\item[] \texttt{file}: Die Datei aus der die Entität geladen werden soll
	\item[]Returns: die geladenen Entität oder \texttt{null} wenn keine Entität des Typs \texttt{T} geladen werden konnte
	
	\item \underline{\texttt{static <T> List<T> loadAllEntities(File dir)}}
	\item[] Lädt alle Entität des Typs \texttt{T} aus den Dateien in dem gegebene Ordner 
	\item[] \texttt{<T>} Entitätentyp der geladen werden soll
	\item[] \texttt{dir}: Der Ordner aus dem die Entität geladen werden soll
	\item[]Returns: eine Liste aller geladener Entitäten des Typs \texttt{T}
	
	\item \underline{\texttt{static <T> List<T> loadAllEntities(List<File> files)}}
	\item[] Lädt alle Entität des Typs \texttt{T} aus den gegebenen Dateien
	\item[] \texttt{<T>} Entitätentyp der geladen werden soll
	\item[] \texttt{files}: Die Liste der Dateien aus denen die Entitäten geladen werden sollen
	\item[]Returns: eine Liste aller geladener Entitäten des Typs \texttt{T}
	
	\item \underline{\texttt{static <T> boolean saveEntity(File file, T entity)}}
	\item[] Speichert die Entität des Typs \texttt{T} in der gegebenen Datei
	\item[] \texttt{<T>} Entitätentyp der gespeichert werden soll
	\item[] \texttt{file}: Die Datei, in welche die Entität gespeichert werden soll
	\item[] \texttt{entity}: Die Entität die gespeichert werden soll
	\item[]Returns: \texttt{true}, wenn das Speichern erfolgreich war, andernfalls \texttt{false}
\end{itemize}

\section{Glossar}
\textbf{Elementare Konfiguration:}
Eine Konfiguration, in der Multikonfiguration deaktiviert ist. Mit den \enquote{einer Konfiguration zugehörigen elementaren Konfigurationen} wird im Falle einer Multikonfiguration die Menge aller elementaren Konfigurationen bezeichnet, in denen der Multikonfigurationsparameter die festgelegte Wertemenge durchläuft. Im Falle einer elementare Konfiguration ist wieder die Konfiguration selbst gemeint.

\end{document}
