\documentclass[parskip=full,11pt]{scrartcl}

\usepackage[sfdefault,light]{roboto}
\usepackage{inconsolata}
\usepackage[ngerman]{babel}

\usepackage[utf8]{inputenc}
\usepackage[T1]{fontenc}

\usepackage{microtype}

\usepackage{csquotes}
\MakeOuterQuote{"}

\usepackage{graphicx}
\usepackage{float}
\usepackage{bm}
\usepackage{amssymb}
\usepackage[hidelinks]{hyperref}
\usepackage[section]{placeins}


\usepackage[T1]{fontenc}
\usepackage[scaled=0.85]{beramono}

\begin{document}
\begin{titlepage}
	\centering
	\vspace*{5cm}
	\includegraphics[width = 0.7\linewidth]{img/logo.png}\par
	{\huge\bfseries Ein Simulator für wiederholte Spiele\par}
	%\vspace{1cm}
	{\Large Implementierungsbericht\par}
	\vspace{2cm}
	{\Large\itshape Sebastian Feurer, Peter Koepernik, Luc Mercatoris,\\Christian Schorr, Pierre Toussing\par}
	\vfill
	{\large \today\par}
\end{titlepage}

\tableofcontents
\pagebreak

\section{Einleitung}
Nach der Spezifikation des Programms in der ersten und dem Entwurf einer Programmarchitekur in der zweiten wurde das Programm in dieser Phase nun implementiert. Der Ablauf und die Ergebnisse der Implementierungsphase sollen hier kurz vorgestellt werden.

Innerhalb der letzten vier Wochen wurden durch insgesamt über X Commits im Versionierungssystem git X Klassen und Schnittstellen in X Paketen implementiert, welche  insgesamt X Java-Quellcode-Zeilen umfassen. Des Weiteren wurden zahlreiche Komponententests erstellt.

Im Laufe der Implemetierung sind zahlreiche verschiedene Bibliotheken und Programme zum Einsatz gekommen, die hier kurz vorgestellt werden sollen.

\subsection{Gradle}
\begin{enumerate}
\item[] \textbf{Ziel:} Automatisierung des Build-Prozesses
\item[] \textbf{Beschreibung:} Gradle ist ein auf Java aufbauendes Build-Management-Automatisierungs-Tool zum automatischen Erstellen eines ausführbaren Anwendungsprogramm aus Java-Quellcode.
\item[] \textbf{Erläuterung:} Zur Automatisierung des Build-Prozesses und zum vereinfachten Dependency-Management bei Verwendung externer Bibliotheken bietet sich der Einsatz eines Build-Tools an. Auf Empfehlung unseres Betreuers hin entschieden wir uns dabei für Gradle.
\end{enumerate}
\subsection{Eclipse und IntelliJ}
\begin{enumerate}
\item[] \textbf{Ziel:} Bereitstellung eines Editors.
\item[] \textbf{Beschreibung:} Eclipse sowie IntelliJ sind Open-Source-Code-Editoren.
\item[] \textbf{Erläuterung:} Ein Editor ist für das Erstellen eines Programms unerlässlich, da er die Organisation des Projekts, das Erstellen von Klassen und Schnittstellen, das Finden von Fehlern und automatische Ausführen von Tests sowie das Kompilieren erleichtert. Je nach bisherigen Erfahrungen entschieden sich die Mitglieder unseres Teams dabei für Eclipse oder IntelliJ.
\end{enumerate}
\subsection{JUnit}
\begin{enumerate}
\item[] \textbf{Ziel:} Testen von Komponenten
\item[] \textbf{Beschreibung:} JUnit ist ein Test-Framework für Java.
\item[] \textbf{Erläuterung:} Für das Testen einzelner Klassen und Interfaces, besonders zur Vermeidung von Regressionen, ist ein Framework nützlich, dass das automatische Ausführen von Komponententests erlaubt.
\end{enumerate}
\subsection{JavaFX}
\begin{enumerate}
\item[] \textbf{Ziel:} Erstellen einer ansprechenden Benutzeroberfläche.
\item[] \textbf{Beschreibung:} JavaFX ist Framework zur Erstellung von Benutzeroberflächen für Java-Applikationen.
\item[] \textbf{Erläuterung:} Als GUI-Frameworks kamen zu Beginn Swing und JavaFX in Frage. Da einer der Entwickler bereits in früheren Projekten Erfahrung mit JavaFX gesammelt hat, haben wir uns für dieses Framework entschieden.
\end{enumerate}
\subsection{ControlsFX}
\begin{enumerate}
\item[] \textbf{Ziel:} Bereitstellung weiterer GUI-Elemente
\item[] \textbf{Beschreibung:} ControlsFX ist eine GUI-Bibliothek, die JavaFX ergänzt.
\item[] \textbf{Erläuterung:} Einige der verwendeten GUI-Elemente sind in JavaFX nicht enthalten und werden von ControlsFX bereitgestellt.
\end{enumerate}
\subsection{JGraphT}
\begin{enumerate}
\item[] \textbf{Ziel:} Bereitstellung eines effizienten Matching-Algorithmus für Graphen
\item[] \textbf{Beschreibung:} JGraphT ist eine Open-Source-Graphbibliothek für Java.
\item[] \textbf{Erläuterung:} In einem der verwendeten Paarbildungsalgorithmen (der \enquote{Paarbildung nach Wunsch}) muss ein Matching auf einem Graphen mit gewichteten Kanten berechnet werden. Da das Entwerfen und Implementieren von Graphenalgorithmen nicht Teil dieses Projekts ist, haben wir dazu auf eine Bibliothek zurückgegriffen.
\end{enumerate}
\subsection{Gson}
\begin{enumerate}
\item[] \textbf{Ziel:} (De-)Serialisierung von Java-Objekten
\item[] \textbf{Beschreibung:} Gson ist eine Open-Source-Java-Bibliothek zum Serialisieren und Deserialisieren von Java-Objekten.
\item[] \textbf{Erläuterung:} Beim Speichern von Simulationsergebnissen wollten wir nicht auf die Java-eigene Serialisierungs-Bibliothek zurückgreifen, da das erzwungen hätte, viele der Klassen im Model serialisierbar zu machen. Das wäre entwurfstechnisch nicht schön, weshalb wir uns entschieden haben, hier auf eine andere Bibliothek zurückzugreifen.
\end{enumerate}
\subsection{Apache Commons Math}
\begin{enumerate}
\item[] \textbf{Ziel:} 
\item[] \textbf{Beschreibung:} Apache Commons Math ist eine Open-Source-Java-Bibliothek, die grundlegende mathematische und statistische Werkzeuge anbietet.
\item[] \textbf{Erläuterung:} Bei der Kapitalinitialisierung von Agenten können diskrete Wahrscheinlichkeitsverteilungen wie die Binomial- oder Poissonverteilung verwendet werden. Für die Generierung von entsprechend verteilten Pseudo-Zufallszahlen haben wir auf diese Bibliothek zurückgegriffen.
\end{enumerate}

\section{Änderungen am Entwurf}
Pakete X und Y hinzugefügt/entfernt
\subsection{Anpassung der Ausgabe}\label{outputmod}

\newpage
\section{Implementierte Funktionalitäten}
Die folgende Tabelle liefert einen Überblick über die im Pflichtenheft beschriebenen Funktionalitäten und deren Umsetzung.

\begin{table}[h]
\centering
\begin{tabular}{l | l | l | l}
\textbf{Nummer} & \textbf{Beschreibung} & \textbf{implementiert?} & \textbf{Bemerkung} \\
\hline
\textbf{M1} & Starten einer Simulation  & \checkmark \\
\textbf{M2} & Ausgabe der Simulationsergebnisse & \checkmark & 1)\\
\textbf{M3} & Abbrechen einer Simulation & \checkmark \\
\textbf{M4} & Festlegung von Simulationsparametern & \checkmark \\
\textbf{K1} & Multikonfiguration &  \checkmark \\
\textbf{K2} & Erstellen eigener Strategien & \checkmark \\
\textbf{K3} & Erstellen eigener Stufenspiele & \checkmark\\
\textbf{K4} & Speichern und Laden von Konfigurationen & \checkmark & 2)\\
\textbf{K5} & Speichern und Laden von Simulationsergebnissen & \checkmark & 2) \\
\textbf{K6} & Starten mehrerer Simulationen & \checkmark \\
\textbf{K7} & Anpassen der Gleichgewichtsbedingung & \checkmark \\
\textbf{K8} & Erweiterte Gruppenfunktionalität & \checkmark & 3) \\
\hline
\textbf{Gesamt} &\textbf{12} & \textbf{12}
\end{tabular}
\caption{Übersicht der implementierten Muss- und Kann-Kriterien}
\end{table}
Wie in der Tabelle zu sehen konnten alle Kriterien erfolgreich umgesetzt werden, dazu jedoch noch folgende Anmerkungen:
\begin{enumerate}
\item[] 1) Die Form der Ausgabe für die Strategie- und Kapitalverteilungen der Agenten wurde angepasst (siehe \ref{outputmod}).
\item[] 2) Neben Konfigurationen und Simulationsergebnissen können auch Gruppen, Populationen, Strategien und Stufenspiele als Dateien exportiert und importiert werden. Zur Erläuterung dieser Begriffe sei auf Pflichtenheft und Entwurfsdokument verwiesen.
\item[] 3) Die Struktur von Gruppen und Segmenten hat sich in der Entwurfsphase durch die Einführung von Populationen geändert (siehe Entwurfsdokument).
\end{enumerate}

Neben den in der Tabelle erwähnten Funktionen konnten wir auch erfolgreich die Funktionalitäten implementieren, die wir in der Entwurfsphase noch zur Spezifikation hinzugefügt haben, also das Konzept von Populationen, erweiterte Möglichkeiten zur Erstellung von Strategien und das Plugin-System.

\section{Schwierigkeiten bei der Implementierung}

\subsection{Erstellung des Multisliders}

\subsection{Das Plugin-System}
wegen gradle

\subsection{Speichern von Simulationsergebnissen}
-> gson

\section{Implementierungsplan}
zeitmanagement etc.

\end{document}