\documentclass[parskip=full,11pt,openany]{scrreprt}

\usepackage[sfdefault,light]{roboto}
\usepackage{inconsolata}
\usepackage[ngerman]{babel}

\usepackage[utf8]{inputenc}
\usepackage[T1]{fontenc}

\usepackage{microtype}

\usepackage{csquotes}
\MakeOuterQuote{"}

\usepackage{graphicx}
\usepackage{float}
\usepackage{bm}
\usepackage{amssymb}
\usepackage[hidelinks]{hyperref}
\usepackage[section]{placeins}
\usepackage{booktabs}

\usepackage{amsmath}

\usepackage{enumitem} 

\usepackage{amssymb}% http://ctan.org/pkg/amssymb
\usepackage{pifont}% http://ctan.org/pkg/pifont
\newcommand{\cmark}{\ding{51}}%
\newcommand{\xmark}{\ding{55}}% 

\usepackage[]{hyperref}
\usepackage{makecell}


%Change enummeration pattern from 1. to 1)
\renewcommand\labelenumi{\theenumi)}

%align lambda terms left, no spacing before or after environment
\newenvironment{nospaceflalign*}
 {\setlength{\abovedisplayskip}{0pt}\setlength{\belowdisplayskip}{0pt}%
  \csname flalign*\endcsname}
 {\csname endflalign*\endcsname\ignorespacesafterend}


\titlehead{\centering\includegraphics[width=6cm]{img/logo.pdf}}
\title{Testbericht}
\subtitle{Wavelength--$\bm{\lambda}$-IDE}
\author{Muhammet Gümüs, Markus Himmel, Marc Huisinga,\\Philip Klemens, Julia Schmid, Jean-Pierre von der Heydt}

\begin{document}
\begin{titlepage}
	\centering
	\vspace*{5cm}
	\includegraphics[width = 0.7\linewidth]{images/logo.png}\par
	{\huge\bfseries Ein Simulator für wiederholte Spiele\par}
	%\vspace{1cm}
	{\Large Implementierungsbericht\par}
	\vspace{2cm}
	{\Large\itshape Sebastian Feurer, Peter Koepernik, Luc Mercatoris,\\Christian Schorr, Pierre Toussing\par}
	\vfill
	{\large \today\par}
\end{titlepage}

\tableofcontents
\pagebreak

\chapter{Überblick}

\section{Einleitung}

In diesem Dokument werden die Ergebnisse der Qualitätssicherungsphase des Loop-Projekts festgehalten.
In dieser Phase wurden neben ausführlichem Testen Performance-Verbesserungen vorgenommen und noch einige neue Features implementiert.

\section{Änderungen in der Applikation}


\section{Abdeckung der Tests im Pflichtenheft}

Die folgende Tabelle greift die im Pflichtenheft definierten Test auf und gibt an, welche davon getestet wurden. Da es sich fast ausschließlich um GUI-Tests handelt, erfolgte das Testen größtenteils manuell.

Einige Testszenarien wurden an Spezifikationsänderungen angepasst. Etwa hat sich die Struktur des Konfigurationsfensters durch das Einführen von Populationen in der Entwurfsphase geändert.

\begin{table}[h]
	\centering
	\begin{tabular}{@{}ll|c|r@{}}
		\toprule
		&\textbf{Test Nr.} & \textbf{Getestet} &\textbf{Anmerkung} \\ 
		\midrule
		\multicolumn{3}{l|}{\small \textsc{\textbf{T1} Grundeinstellungen}} \\ 
		&T1.1 - T1.4 & \cmark & \\
		&T1.5 & (\xmark) & \makecell{Einstellung wurde mit dem Einführen\\ von Populationen entfernt}\\
		&T1.6 & \cmark & Der Slider wurde durch ein Textfeld ersetzt\\
		&T1.7 & \cmark & \\
		&T1.8 & \cmark & \\
		&T1.9 - T1.10 & (\xmark) & \makecell{Einstellungen wurden mit dem Einführen\\ von Populationen entfernt}\\
		&T1.11 & \cmark & \\ 
		\multicolumn{4}{l}{\small \textsc{\textbf{T2} Simulationen starten und abbrechen, Ergebnisanzeige}}\\ 
		&T2.1 & \cmark & \\
		&T2.2 - T2.3 & \cmark & Der Knopf zum Abbrechen wurde verschoben\\
		&T2.4 - T2.6 & \cmark & \\
		\multicolumn{4}{l}{\small \textsc{\textbf{T3} Konfigurationen und Simulationen speichern und laden}}\\ 	
		&T3.1 - T3.4 & \cmark & \makecell{Konfigurationen werden nun vom Hauptfenster aus\\ geladen und gespeichert}\\
		&T3.5 - T3.9 & \cmark & \\
		\multicolumn{4}{l}{\small \textsc{\textbf{T4} Gruppen- und Segmenteinteilung}}\\ 
		&T4.1 - T4.8 & (\cmark) & \makecell{Die Gruppenerstellung wurde mit dem Einführen von\\ Populationen überarbeitet und entsprechend getestet} \\
		\multicolumn{3}{l|}{\small \textsc{\textbf{T5} Stufenspiel erstellen}}\\ 
		&T5.1 - T5.2 & \cmark & \\
		\multicolumn{4}{l}{\small \textsc{\textbf{T6} Festlegung des Algorithmus zur Paarbildung}}\\ 
		&T6.1 - T6.2 & \cmark & \\
		\multicolumn{4}{l}{\small \textsc{\textbf{T7} Festlegung der Erfolgsquantifizierung}}\\ 
		&T7.1 - T7.3 & \cmark & \\
		\multicolumn{4}{l}{\small \textsc{\textbf{T8} Festlegung des Adaptionsmechanismus}}\\ 
		&T8.1 - T8.3 & \cmark & \\
		\multicolumn{4}{l}{\small \textsc{\textbf{T9} Gleichgewichtskriterium und Schranke für Adaptionsschritte festlegen}}\\ 
		&T9.1 - T9.4 & \cmark & \\
		\multicolumn{3}{l|}{\small \textsc{\textbf{T10} Multikonfiguration}}\\ 
		&T10.1, T10.3 & \cmark & \makecell{Die Einstellung des Multi-Parameters erfolgt nun\\ über ein dediziertes Dropdown-Menü}\\
		&T10.2, T10.4 - T10.6 & \cmark & \\
		\multicolumn{3}{l|}{\small \textsc{\textbf{T11} Strategie erstellen}}\\
		&T11.1 - T11.3 & \cmark & \\
		\multicolumn{3}{l|}{\small \textsc{\textbf{T12} Fehlerbehandlung}}\\ 
		&T12.1, T12.3 - T12.7 & (\xmark) & \makecell{Einstellungen wurden mit dem Einführen\\ von Populationen entfernt}\\
		&T12.2 & (\cmark) & \makecell{Statt einer Autokorrektur wird eine fehlerhafte\\ Eingabe entsprechend markiert}\\
		\bottomrule
	\end{tabular}
	\caption{Übersicht zur Abdeckung der Test aus dem Pflichtenheft}
\end{table}

\section{Statistiken}

X testklassen, X tests, X zeilen testcode

prozent testabdeckung

\end{document}
